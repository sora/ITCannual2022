\subsection{研究報告(石川グループ研究室)}

%  センサやロボットはもちろんのこと、社会・心理現象等も含めて、現実の物理世界は、原則的に並列かつリアルタイムの演算構造を有している。その構造と同等の構造を工学的に実現することは、現実世界の理解を促すばかりでなく、応用上の様々な利点をもたらし、従来のシステムをはるかに凌駕する性能を生み出すことができ、結果として、まったく新しい情報システムを構築することが可能となる。本研究室では、特にセンサ情報処理における並列処理と高速・リアルタイム性を高度に示現する研究として、以下4つの分野での研究を行っている。また、新規産業分野開拓にも力を注ぎ、研究成果の技術移転,共同研究,事業化等を様々な形で積極的に推進している。

% 五感の工学的再構成を目指したセンサフュージョンの研究では、理論並びにシステムアーキテクチャの構築とその高速知能ロボットの開発、その応用としての新規タスクの実現、特に、視・触覚センサによるセンサ情報に基づく人間機械協調システムの開発を行っている。

% ダイナミックビジョンシステムの研究では、高速ビジョンや動的光学系に基づき運動対象の情報を適応的に取得する基礎技術の開発、特に、高速光軸制御や可変形状光学系の技術開発やトラッキング撮像に関する応用システムの開発を行っている。

% 高速三次元形状計測や高速質感計測など、並列処理に基づく高速画像処理技術 (理論、アルゴリズム、デバイス) 開発とその応用システムの実現を目指すシステムビジョンデザインの研究では、特に高速画像処理システムの開発、高速性を利用した新しい価値を創造する応用システムの開発を行っている。

% 実世界における新たな知覚補助技術並びにそれに基づく新しい対話の形の創出を目指すアクティブパーセプション技術の構築とその応用に関する研究では、特に各種高速化技術を用いた能動計測や能動認識を利用した革新的情報環境・ヒューマンインタフェイスの開発を行っている。

