% \begin{雑誌論文}{1}
% \bibitem{01石川正俊01}
% 川原大宙,妹尾拓,石井抱,平野正浩,岸則政,石川正俊:輪郭情報に基づくテンプレートマッチングを用いた重畳車両の高速トラッキング,計測自動制御学会論文集,58巻,1号,pp.21-30,2022.

% \bibitem{01石川正俊02}
% 小山佳祐,堀邊隆介,安田博,万偉偉,原田研介,石川正俊:ワンボード・USB給電タイプの高速・高精度近接覚センサの開発とプリグラスプ制御の解析,日本ロボット学会誌,Vol.39,No.9,pp.862-865,2021.

% \bibitem{01石川正俊03}
% Masahiro Hirano,YujiYamakawa,Taku Senoo and Masatoshi Ishikawa:An acceleration method for correlation-based high-speed object tracking,Measurement Sensors,Vol.18,Article No.100258,2021.

% \bibitem{01石川正俊04}
% Masahiko Yasui,Yoshihiro Watanabe,and Masatoshi Ishikawa:Wide viewing angle with a downsized system in projection-type integral photography by using curved mirrors,Optics Express,Vol.29,Issue8,pp.12066-12080,2021.

% \bibitem{01石川正俊05}
% Ruimin Cao,Jian Fu,Hui Yang,Lihui Wang,and Masatoshi Ishikawa:Robust optical axis control of monocular active gazing based on pan-tilt mirrors for high dynamic targets,Optics Express,Vol.29,No.24,pp.40214-40230,2021.

% \end{雑誌論文}

% \begin{査読付}{1}

% \bibitem{01石川正俊06}
% Hiromichi Kawahara,Taku Senoo,Idaku Ishii,Masahiro Hirano,Norimasa Kishi and Masatoshi Ishikawa:High-speed tracking for overlapped vehicles using Instance Segmentation and contour deformation,2022 IEEE/SICE International Symposium on System Integration (SII2022),Proceedings,pp.730-735,2022.

% \bibitem{01石川正俊07}
% Masahiro Hirano,Yuji Yamakawa,Taku Senoo,Norimasa Kishi,Masatoshi Ishikawa:Multiple Scale Aggregation with Patch Multiplexing for High-speed Inter-vehicle Distance Estimation,IEEE Intelligent Vehicles Symposium (IV),Proceedings,pp.1436-1443,2021.

% \bibitem{01石川正俊08}
% Hirofumi Sumi,Hironari Takehara,Jun Ohta,and Masatoshi Ishikawa:Advanced Multi-NIR Spectral Image Sensor with Optimized Vision Sensing System and Its Impact on Innovative Applications,2021 Symposium on VLSI Technology ,2021 Symposium on VLSI Technology Digest of Technical Papers,JFS4-8,pp.1-2,2021.

% \bibitem{01石川正俊09}
% Masahiko Yasui,Yoshihiro Watanabe,and Masatoshi Ishikawa:Dynamic and Occlusion-Robust Light Field Illumination,ACM SIGGRAPH ASIA 2021 Posters (SIGGRAPH ASIA 2021),Proceedings,Article No.35,pp.1–2,2021.

% \end{査読付}

% \begin{発表}{1}

% \bibitem{01石川正俊10}
% 石川正俊:高速ビジョンを用いた高速知能ロボット,ロボット,No.263,pp.56-58 ,2021.

% \bibitem{01石川正俊11}
% 石川正俊:情報科学技術の構造と情報教育,IDE 現代の高等教育,2021年8-9月号,No.633,pp.9-13,2021.

% \bibitem{01石川正俊12}
% 谷内田尚司,並木重哲,小川拓也,細井利憲,石川正俊:高速カメラ物体認識技術を用いた錠剤外観検査装置,製剤機械技術学会誌,Vol.30,No.4,pp.35-40,2021.

% \end{発表}

% \begin{報道}{1}
% \bibitem{01石川正俊13}
% 石川正俊:「ロボット」100年で次へ 東大特任教授・石川正俊氏に聞く 人のはるかに先を行く 機械性能極限まで発揮,電波新聞,令和3年7月8日.

% \bibitem{01石川正俊14}
% 石川正俊:「高速反応や自律航行 ロボット研究進む 東大でオンライン公開講座」,電波新聞,令和3年6月17日.

% \bibitem{01石川正俊15}
% 石川グループ研究室:MBS毎日,日曜日の初耳学,「VarioLight2,ダイナミックプロジェクションマッピング」,令和3年9月13日.

% \bibitem{01石川正俊16}
% 石川グループ研究室:テレビ東京,日経ニュースプラス9,「ElaMorph projection,VarioLight2,高速トンネル検査」,令和3年6月8日.

% \bibitem{01石川正俊17}
% 石川グループ研究室:TBS,あさチャン!,「ジャンケンロボット,高速道路トンネル検査」,令和3年6月28日.

% \end{報道}






