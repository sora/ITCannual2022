\subsection{研究報告(黄 守仁)}

%  本年度は主に人間協調ロボットの提案・実装、生産システムの知能化を目指した動的補償ロボットの開発、高速三次元計測によるロボット制御の検討を中心に研究活動を行った。

% 人間協調ロボットに関しては、人間の認知能力とロボットの高速・高精度な動作を相互補完的に組み合わせることを目指して、力覚提示によるヒューマンロボットインタラクションと人間の両腕同期運動現象を統合する手法の提案、タスク検証および被験者実験を行った。人間の両腕同期運動現象を定量的に分析し、得られた分析結果に基づいて適切な実験システムを構築した。10人ほどの被験者実験を実施し、両腕同期運動による力学提示を両腕の間に伝達する効果を検証した。また、企業との共同研究においても、人間ロボット共存型知能生産の実現を目指して、塗布など実応用を対象とした人間協調ロボットの設計・開発を行った。

% 次に、前年度に新規開発した3自由度動的補償モジュールを商用の産業ロボットに搭載し、従来の教示作業を必要としない産業用ロボットの知能化・自律化に向けて研究を推進した。また、企業との共同研究課題として、高速三次元計測を動的補償ロボット制御に統合するシステムの設計およびシミュレーションモデル構築などにも取り組んできた。その他に、前年度に開発した高周波電気刺激装置を用いて高周波外部フィードバック情報による人間の上腕に対する電気刺激制御の検証実験にも着手した。

