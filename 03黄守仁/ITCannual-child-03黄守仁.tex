\subsection{知能ロボットおよび人間機械協調の実現に向けた研究開発(黄守仁)}

 本年度は主に生産システムの知能化を目指した動的補償ロボットの開発と評価実験、無拘束の視線入力などに基づく人間・ロボットの協調の提案および実装、高速三次元計測によるロボット制御などの研究内容で研究活動を行った。

生産システムの知能化を目指した(企業との)共同研究において、前年度の研究を継続し、一般的な産業用ロボットが速度・精度・不確定要素に対する適応能力の三者を同時に成立させることが難しい現状に対して、高帯域でのセンシング・動作によるローカル誤差吸収と低い帯域でのグローバル計画動作を並列に行う動的補償手法を提案し、コンベア上を流れている事前情報(置く位置・姿勢、塗布形状などの情報を指す)のない部品に対する高精度な塗布作業の実現を可能とするロボットを開発した。初期の評価実験で得られた結果を計測自動制御学会システムインテグレーション部門SI2022にて発表を行い、研究成果が認められ、優秀講演賞を受賞した。

次に、次世代サイボーグシステムの実現に向けて、無拘束の視線入力とロボットによるセンシング支援・動作支援を統合する人間機械協調手法を提案し、ROS環境での実装および予備実験を行った。室内の日常生活を想定し、人間の視線入力とロボットの視覚センシング支援および動作支援の応用場面を検討した。特に、南デンマーク大学からの博士課程学生が協力研究補助員として短期訪問をしている間、システム実装に向けてとても有意義な交流を行った。

また、企業との共同研究課題として、高速三次元計測を動的補償ロボット制御に統合するシステムの設計およびシミュレーションモデル構築など、前年度の研究活動も継続して取り組んでいる。
