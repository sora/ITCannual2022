\subsection{研究報告(金 賢梧)}

% 高速カメラネットワークの実時間同期制御についての研究

% (背景)

% 人波のトラッキングはセキュリティー向けのカメラネットワーク領域で重要なテーマの一つである。従来のカメラネットワークを用いたトラッキングでは、比較的遅い撮像速度が原因で、計算負荷の大きい画像処理や推測基盤のデータ解析などが必須であった。本研究では1秒に1000枚といった高速な撮像および画像処理能力を有するカメラネットワークを構築することで、推測ではなく実際計測による低負荷で剛健な人波トラッキングの実現を目標とする。本年度はその基礎研究として高速カメラネットワークの同期精度を評価する手法についての研究を行なった。

% (研究内容と成果)

% 高速カメラネットワークにおける同期撮像精度を画像情報から直接求めるアルゴリズムを提案し、直線的に振動するレーザー光源を用いた評価実験を行なった。その結果、サブフレーム(1ms)以下の定量的な同期精度評価が可能であることを確認した。カメラネットワークの撮像同期を外部の計測装置を用いず、取得した画像情報だけを用いて評価する従来の手法に比べて、サブフレーム以下の高い計測精度が得られたことや、輝度情報ではない位置情報の移動量を用いることで周辺環境に対するロバスト性や活用性を高めたことが研究成果と考えられる。カメラネットワークの撮像同期を、画像情報だけを用いて簡単に、精度良く求められることは、実用的な面で活用性が高い。今後高速カメラネットワークによる3次元形状・位置計測においての計測結果を改善するためのシステム評価および同期性能向上に寄与できると考えられる。