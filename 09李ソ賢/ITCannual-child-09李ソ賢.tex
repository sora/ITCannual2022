\subsection{研究報告(李ソ賢)}

%  本研究者はディープラーニングアリゴリズムを用いて生体イメージングデータから有意味な情報を取り出す研究に取り組んできた。今年度は癌細胞の顕微鏡イメージから転移しやすい癌と比較的に転移しない癌をそのイメージだけで区別する深層ネットワークを作成することを目標とした。対象物質としてはヒト乳癌細胞であるKPL4を用い、そのwild typeの細胞を転移しないモデル、また、PAR1という膜タンパク質が発現されている細胞を転移しやすいモデルとして設定し、トレーニングデータセットを作成した。深層学習を適用する際には、汎用性を持つようにすでに構築されているモデルをそのまま使う方法と、最初から一つの目的を持つように設計する方法の二つがある。

% 本研究者は癌細胞の転移性を区別するためその二つの方法を比較し、実際の癌イメージングデータに対してより正確な結果を出せる深層学習方法を提案した。この成果は人工知能関連の国際会議である第4回International Conference on Artificial Intelligence in Information and Communicationで発表を行い、様々な研究者たちから良い反応を得ることができた。
