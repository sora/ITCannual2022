\subsection{野生動物ワイヤレスセンサネットワーク実証実験基盤構築に向けた研究(川瀬 純也)}
%  本研究室では、野生動物装着型ワイヤレスセンサーネットワーク(以下:野生動物WSN)機構による自然環境でのデータ収集手法の開発と、それによって得られるデータの解析手法についての研究を行っている。これらを実現するためにはいくつかの課題が存在している。そのひとつは、野生動物に装着可能なサイズ・重量の制限からモバイル端末のバッテリーが限られ、センサーを十分に稼働させることができずにデータ収集に支障をきたす点である。
 
% これを解決するため、野生動物の群れと個体の個性に着目し、野生動物WSNを構築するために必要な機能を群れ内に分散させる機構の開発を行っている。通例の群れを成す動物を対象とした野生動物WSNでは、群れ内の野生動物1個体にモバイル端末を装着し、その1つに機能を集約する。しかし、群れ内の複数の個体に端末を装着し、各々に機能を分散させることで、ひとつひとつのバッテリー消費量を低減させることができる。また、個体の個性に合わせ、活動的な個体には他の群れとの接触を判定しマルチホップ通信を開始する機能を持たせたり、群れの中心であり生存率が高いと考えられる個体にはデータストアの機能を持たせたりするなど、各々の野生動物の特徴に合わせて効率的なデータ収集と回収の実現を目指す。
 
% 現在、モバイル端末の試作品を作成し、今後評価実験等を行う予定である。この研究の一部は『通信スケジュールが不確定な野生動物IoT網における効率的かつ精確なデータ共有手法の開発』と題して、公益財団法人GMOインターネット財団による2021年度GMO研究助成制度の助成を受け行われた。コロナ禍により研究活動が停滞したため、2022年度も引き続き助成を受ける予定である。
 
% 今後は、飼育動物や放牧場などを活用した野生動物WSNの評価実験環境を目指し、将来的にはより自然環境に近い実証実験環境の整備などを視野に入れて研究を進めていく。
