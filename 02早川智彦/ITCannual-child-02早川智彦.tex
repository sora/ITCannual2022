\subsection{研究報告(早川 智彦)}

%  2021年度は主に1.高速画像処理技術によるモーションブラー補償、2.サーモカメラによる温度情報計測の定量化、3.人間の知覚情報の定量化の研究を実施した。全体を通した研究成果として、2件の雑誌論文(査読付)、3件の解説論文と1件の雑誌以外の査読付き論文を投稿し、4件の発表を行った。

% 1.高速画像処理技術によるモーションブラー補償

% インフラ点検に関する表面変状の撮像技術として、デフォーマブルミラーを用いたフォーカス調整とモーションブラー補償を同時に行う手法を提案することで、広い被写界深度と高速な移動の両方に対応する撮像環境に対応可能な技術を確立した。この成果を国際論文誌に投稿し、採択に至った。また、インフラ点検の関連技術をまとめ、複数の解説論文投稿や招待講演を行うことにより、技術の社会実装が円滑に進むよう努めた。

% 2.サーモカメラによる温度情報計測の定量化

% インフラ点検に関する内部変状の撮像技術として、サーモカメラによる撮像は有効であるが、同時に可視光画像の認識を行うことが困難であるため、両者に対応するマーカーを開発した。複数の素材の温度特性や可視光反射率を基に、両者のマーカーとしての見え方が最適化される素材の組み合わせを探求し、結果絶縁体と非絶縁体である黒い和紙と銅箔の組み合わせが最適であることを発見した。この成果を国際論文誌に投稿し、採択に至った。

% 3.人間の知覚情報の定量化

% 低遅延な映像のユーザへの影響を検証するため、高速カメラと高速プロジェクタを用い、遅延時間と対象の移動速度に応じてパフォーマンスが低下することを確認し、国内学会にて発表した。
