% \begin{招待論文}{1}
% \bibitem{02早川智彦01}
% 早川智彦,石川正俊,亀岡弘之:時速100km走行でのトンネル覆工コンクリート高解像度変状検出手法,建設機械施工,vol.73,no.8,pp.19-23,2021.

% \bibitem{02早川智彦02}
% 早川智彦,望戸雄史,石川正俊,大西偉允,亀岡弘之:【大臣賞】時速100km走行での覆工コンクリート高解像度変状検出手法,土木施工,vol.62,no.7,p.146,2021.

% \end{招待論文}

% \begin{雑誌論文}{1}
% \bibitem{02早川智彦03}
% Kenichi Murakami,Tomohiko Hayakawa,and Masatoshi Ishikawa: Hybrid surface measuring system for motion-blur compensation and focus adjustment using a deformable mirror,Applied Optics,vol.61,Issue2,pp.429-438,2022.

% \bibitem{02早川智彦04}
% Yuki Kubota,Yushan Ke,Tomohiko Hayakawa,Yushi Moko,and Masatoshi Ishikawa:Optimal Material Search for Infrared Markers under Non-Heating and Heating Conditions,Sensors,Vol.21,Issue 19,Article No.6527,pp.1-17,2021.

% \end{雑誌論文}

% \begin{査読付}{1}
% \bibitem{02早川智彦05}
% Yuki Kubota,Tomohiko Hayakawa,Osamu Fukayama,and Masatoshi Ishikawa:Sequential estimation of psychophysical parameters based on the paired comparisons,2022 IEEE/SICE International Symposium on System Integration (SII 2022),pp.150-154,2022.

% \bibitem{02早川智彦06}
% Ke Yushan,Yushi Moko,Yuka Hiruma,Tomohiko Hayakawa,and Masatoshi Ishikawa:Silk printed retroreflective markers for infrastructure maintenance vehicles in tunnels,SPIE Smart Structures + NDE On Demand,2022 (accepted).

% \end{査読付}

% \begin{発表}{1}
% \bibitem{02早川智彦07}
% 栃岡 陽麻里,早川 智彦,石川 正俊:身体感覚と視覚情報にずれが生じる低遅延没入環境におけるターゲットの加速度がユーザへ与える影響,第26回日本バーチャルリアリティ学会大会 (VRSJ2021),論文集,3B2-3,2021.

% \end{発表}
