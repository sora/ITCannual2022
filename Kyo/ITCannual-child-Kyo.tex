\subsection{時空間データインテリジェンス(姜 仁河)}
本節では2022年度の姜仁河の研究活動について報告する。近年、都市のスマート管理、スマートシティは新しい科学技術分野として各国の学術界、産業界および各国政府から非常に重視されている。モバイルデータ、IoTセンサデータ、衛星画像、交通プローブデータ、災害データなどダイナミックなリアルタイム時空間ビッグデータが入手可能な環境が急速に整いつつあり、健康や医療サービスデータ、購買履歴データ、物流・商流などの経済データも積極的に活用されている。これらのデータを統合した形で人々や企業の活動、交通・物流・商流から都市の拡大・環境変化、社会経済システムの変質・変動までを包含するデジタル社会空間のあらゆる課題を解決する。これを目的にして、引き続き2022年度、私は時空間データインテリジェンスについて研究し、主に深層学習技術に基づき、群衆密度、新型コロナ感染者数、人のモビリティ、汎用時系列向けの高精度予測モデルを開発してきた。関連成果はAI・データサイエンス分野のトップカンファレンスAAAI2023、ECMLPKDD2022、BigData2022及びトップジャーナルIEEE TKDE2022にて発表された。詳細は下記の通りにまとめた。
\cite{JIANG1-1,JIANG1-2,JIANG1-3,JIANG1-4,JIANG1-5,JIANG1-6,JIANG1-7}
\cite{JIANG2-1,JIANG2-2,JIANG2-3,JIANG2-4,JIANG2-5,JIANG2-6,JIANG2-7,JIANG2-8}

