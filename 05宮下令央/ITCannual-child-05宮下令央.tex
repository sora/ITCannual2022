\subsection{研究報告(宮下 令央)}

 % 本年度は高速三次元形状計測手法の開発およびダイナミックプロジェクションマッピング技術の発表、研究周知活動を行った。

 % 高速形状計測手法のひとつとして、以前より高速法線計測システムおよび独自の法線画像処理技術の開発を行っており、本年度は法線特徴量のバイナリ化と並列計算による高速化を達成し、マーカーレスでリアルタイムに対象をトラッキングする技術について論文発表を行った。また、この高速法線計測システムと高速距離計測システムを統合し、距離と法線を単一フレームで同時高速計測するシステムを構築した。両者は高精度に計測できる空間周波数帯が異なるため、数理的に双方の長所を取り入れ、高密度かつ高精度かつ高速な三次元形状計測を実現した。

 % さらに、基盤技術となる高速三次元形状計測について、システム面と数理面から効率化を図り、従来手法よりもさらに高速かつ高密度の三次元形状計測を実現する構造化光パターンおよび画像処理アルゴリズムの開発を行った。本技術は三次元形状計測という汎用的な機能を高い性能を実現できるため、今後は本技術の洗練と詳細な評価を行い、前述した距離と法線を融合する三次元形状計測システムを含めた幅広いシステムに応用していく予定である。

 % また、ダイナミックプロジェクションマッピング技術によって運動を伴う対象のテクスチャおよび弾性を疑似的に変化させるシステムを構築し、国際学会におけるデモや報道において周知を行った。研究会や講演会の運営や査読を通して学会への貢献も引き続き行っている。
