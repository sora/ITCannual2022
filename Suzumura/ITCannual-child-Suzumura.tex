% %半ページから1ページが文量

\subsection{大規模グラフニューラルネットワークの理論と応用(鈴村 豊太郎)}

本節では2022年度の鈴村豊太郎の研究活動について報告する。 鈴村は、 グラフ構造に対するニューラルネットワークを用いた表現学習 Graph Neural Network (以下、GNNと呼ぶ)の基礎研究及び応用研究に取り組んでいる。 グラフ構造は、 ノードと、 ノード同士を接続するエッジから構成されるデータ構造である。 インターネット上における社会ネットワーク、 購買行動、 サプライチェーン、 金融における決済データ、 交通ネットワーク、 蛋白質相互作用・神経活動・DNAシーケンス配列内の依存性、 物質の分子構造、 人間の骨格ネットワーク、 概念の関係性を表現した知識グラフなど、 グラフ構造として表現できる応用先は枚挙に暇がない。
\par
当該研究領域における研究として、時系列・動的に変化する大規模グラフに対するGNNモデルの研究を行った。
動的グラフに対応するGNNモデルはすでに数多く提案されているが、いずれも短期的なデータの変化しか考慮されておらず、実世界で扱われている長期的なグラフデータでは長期的なコンテキストを捉えることができない問題が潜在的に存在していた。この問題に対して、時間幅が非常に長いグラフデータの性質も捉えることができる Spectral Waveletを提案した(AAAI 2023 \cite{aaai-deft}, TLMR \cite{feta})。
%知識グラフに関する研究成果はWWW 2023 \cite{kg-kp}
% 当該研究領域における研究として、時系列・動的に変化する大規模グラフに対するGNNモデルの研究を行った。実世界では時間幅が非常に長いデータを扱うこともあるが既存の動的グラフへのGNNの研究ではそのような点を考慮していない。この問題に対して、学習可能な Spectral Waveletを提案し、AAAI 2023 \cite{aaai-deft}, WWW 2023 \cite{deft}、 TLMR \cite{feta}に採択された。

GNNに関する応用研究も進めている。金融領域においては不正検出に関するGNNモデルの検証を行い、取引ネットワークをヘテロジニアスなグラフ構造に拡張することによりモデル性能の向上を達成した \cite{eth-gnn}。マテリアルズ・インフォマティクスの分野においては、情報基盤センターの芝隼人先生とガラス物質の形成過程モデルに対して高精度なGNNモデルを提案し\cite{botan}、また当該分野における本質的な問題に対する手法として、インバランスなデータの問題を解消するための手法 \cite{xsig-limin}および外挿のためのモデル構築を行った\cite{xsig-takashige}。E-Commerceの領域においては知識グラフを用いた商品推薦手法を提案した (SIGIR 2023 \cite{sigir})。

また、理論モデルの実世界への検証と応用サイドから意味のある研究テーマを発掘するため、企業との共同研究とも進めている。まず、自動車の走行軌跡データから次の位置や経路を予測し、ロケーションリコメンデーションなどに応用するための手法をトヨタ自動車と探求している。走行軌跡データは緯度・経度及び時刻のシーケンスデータとなるが、それを用いると運転行動パターンを捉えることができる。
% まずシーケンスデータからグラフ構造を構築し、そのグラフ構造からGraphormerというニューラルネットワークモデルを走行軌跡データのパターンを捉えられるようなニューラルネットワークのモデルを提案した。
この行動パターンを捉えるために、シーケンスデータをグラフ構造として表現し、Graphormerをベースにした新たなモデルを提案し、他の既存手法よりも高い精度でパターンを予測することを確認した (ECML-PKDD \cite{stgtrans} 査読中)。
% この新たなモデルを他の既存手法と比較し、より高い精度で走行パターンを予測する事を確認した。本研究の成果を ECML-PKDD \cite{stgtrans}に提出した。
来年度はモビリティにおける様々な領域に応用できるように、走行軌跡データや実世界の地図データなどから事前学習モデルを構築する予定である。また、その他に都市全体の二酸化炭素排出量を抑制するために交通流を分散するための手法をこれらの事前学習モデルと深層強化学習を用いて設計・実装する予定である。

また、エス・エム・エス社との共同研究では、介護や医療領域における人材紹介の推薦システムに関する研究を行った。超高齢化社会に突入する中、介護や医療領域における人材不足は深刻であり、より精度の高い人材マッチングが不可欠である。この問題に対して、深層強化学習を用いた人材マッチング数の最適化手法を提案し、特定の求職者・事業者側に偏ってしまう従来の推薦・マッチング手法に対して、偏りを解消できることを確認した (人工知能学会\cite{sms} 6月発表予定)。来年度に関しては更に実データでの検証を進め、企業側での要望を取り入れ、実ビジネスが持つ制約条件を取り入れた最適化モデルを提案していく予定である。また、モデルにおいて求職者と求人側での動的な二部グラフの関係性及び知識グラフを用いてより精度高いモデルを構築していく予定である。

 日本経済新聞社(以下、日経)との共同研究も2022年10月から開始している。日経ではニュースサイトにおける記事に対してより高度な機械学習による推薦システムを目指しており、2022年度は日経側での問題設定やデータの理解を図り、研究テーマの設定に主に取り組んだ。2023年度は推薦問題や広告配信など様々な領域に応用できるように、ユーザーの行動モデルを統一的に表現する事前学習モデルを構築するべく、自己教師付き学習 (Self-Supervised Learning)を用いた手法を設計・実装する予定である。

 また、GNNの概要と最新研究動向に関する記事を人工知能学会誌\cite{jsai-gnn}に寄稿し、Federated Learning(連合学習)の英語書籍向けに Federated Learningを用いた金融不正検知に関する手法を執筆した\cite{fl-book}。mdxプロジェクトに関する第一弾の国際学会論文として IEEE CBDCom\cite{mdx}にて論文発表を行った。雑誌、査読付き論文、招待講演等のリストは以下の通りである。


% How Expressive are Transformers in Spectral Domain for Graphs? \cite{feta}
% Learnable Spectral Wavelets on Dynamic Graphs to Capture Global Interactions \cite{deft}
% Can Persistent Homology provide an efficient alternative for Evaluation of Knowledge Graph Completion Methods? \cite{kg-kp}


% Spatio-Temporal Meta-Graph Learning for Traffic Forecasting \cite{megacrn}

% Ethereum Fraud Detection with Heterogeneous Graph Neural Networks \cite{eth-gnn}

% Federated Learning for Collaborative Financial Crimes Detection \cite{fl-book}



%  本節では2021 年度の鈴村豊太郎の研究活動について報告する。 2021年4月に本学に着任し、 グラフ構造に関するニューラルネットワークを用いた表現学習 Graph Neural Network (以下、GNNと呼ぶ)の基礎研究及びその様々な応用研究に取り組んでいる。 グラフ構造は、 ノードと、 ノード同士を接続するエッジから構成されるデータ構造である。 インターネット上における社会ネットワーク、 購買行動、 サプライチェーン、 金融における決済データ、 交通ネットワーク、 蛋白質相互作用・神経活動・DNAシーケンス配列内の依存性、 物質の分子構造、 人間の骨格ネットワーク、 概念の関係性を表現した知識グラフなど、 グラフ構造として表現できる応用先は枚挙に暇がない。
% \par
% 当該研究領域において、時系列・動的に変化する大規模グラフに対するGNNモデルの研究を行った。分散計算環境においてスケールするGNNモデルを提唱し、その成果は高性能計算分野におけるトップカンファレンスSC2021\cite{suzumura-sc2021}に採択された。 また、金融領域における不正検知手法として、TransformerアーキテクチャをベースにしたGNN手法を提案し、国際会議 IEEE SMDS 2021\cite{suzumura-smds21}に採択された。 また、 GNNに関する招待講演\cite{suzumura-canon2021}を行った。

% これらの研究に続いて、推薦システムへのGNNモデルに関する研究を開始している。実データ・実問題に基づいた、社会実装を見据えた研究を進めるべく、医療・介護領域における人材推薦としてエス・エム・エス社、自動車における経路推薦としてトヨタ社と共同研究を2023年4月から本格的に開始する。また、国立研究開発法人物質・材料研究機構NIMSが主導する「マテリアル先端リサーチインフラ」プロジェクトの本学拠点の一貫で、材料情報科学 Materials Informaticsへの研究も開始している。
%  データ科学・データ利活用のためのクラウド基盤 mdx プロジェクトにおいて、 今年度は 課金付き運用開始に向けたシステム拡張、スポットVM、データ共有機構(Platform-as-a-Service)に向けた設計を進めた。 また、 mdxに関する講演活動を国内外において行った\cite{suzumura-axies2021、suzumura-nanotec2021、 suzumura-nci2021}。 mdxの論文においては、国際会議IEEE IC2E2022(10th IEEE International Conference on Cloud Engineering) に2022年3月末に投稿した。 
% %\cite{suzumura-mdx2022}においても論文を公開した。

