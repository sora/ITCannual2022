% %半ページから1ページが文量

\subsection{大規模グラフニューラルネットワークと様々な実社会問題への応用(鈴村 豊太郎)}


How Expressive are Transformers in Spectral Domain for Graphs? \cite{feta}
Learnable Spectral Wavelets on Dynamic Graphs to Capture Global Interactions \cite{deft}
Can Persistent Homology provide an efficient alternative for Evaluation of Knowledge Graph Completion Methods? \cite{kg-kp}

Guaranteeing Service Level Agreements for Triangle Counting via Observation-based Admission Control Algorithm \cite{gqsm}

mdx: A Cloud Platform for Supporting Data Science and Cross-Disciplinary Research Collaborations \cite{mdx}

Spatio-Temporal Meta-Graph Learning for Traffic Forecasting \cite{megacrn}

Ethereum Fraud Detection with Heterogeneous Graph Neural Networks \cite{eth-gnn}

Federated Learning for Collaborative Financial Crimes Detection \cite{fl-book}



%  本節では2021 年度の鈴村豊太郎の研究活動について報告する. 2021年4月に本学に着任し, グラフ構造に関するニューラルネットワークを用いた表現学習 Graph Neural Network (以下、GNNと呼ぶ)の基礎研究及びその様々な応用研究に取り組んでいる. グラフ構造は, ノードと, ノード同士を接続するエッジから構成されるデータ構造である. インターネット上における社会ネットワーク, 購買行動, サプライチェーン, 金融における決済データ, 交通ネットワーク, 蛋白質相互作用・神経活動・DNAシーケンス配列内の依存性, 物質の分子構造, 人間の骨格ネットワーク, 概念の関係性を表現した知識グラフなど, グラフ構造として表現できる応用先は枚挙に暇がない。
% \par
% 当該研究領域において、時系列・動的に変化する大規模グラフに対するGNNモデルの研究を行った。分散計算環境においてスケールするGNNモデルを提唱し、その成果は高性能計算分野におけるトップカンファレンスSC2021\cite{suzumura-sc2021}に採択された. また、金融領域における不正検知手法として、TransformerアーキテクチャをベースにしたGNN手法を提案し、国際会議 IEEE SMDS 2021\cite{suzumura-smds21}に採択された. また, GNNに関する招待講演\cite{suzumura-canon2021}を行った。

% これらの研究に続いて、推薦システムへのGNNモデルに関する研究を開始している。実データ・実問題に基づいた、社会実装を見据えた研究を進めるべく、医療・介護領域における人材推薦としてエス・エム・エス社、自動車における経路推薦としてトヨタ社と共同研究を2023年4月から本格的に開始する。また、国立研究開発法人物質・材料研究機構NIMSが主導する「マテリアル先端リサーチインフラ」プロジェクトの本学拠点の一貫で、材料情報科学 Materials Informaticsへの研究も開始している。
%  データ科学・データ利活用のためのクラウド基盤 mdx プロジェクトにおいて, 今年度は 課金付き運用開始に向けたシステム拡張,スポットVM、データ共有機構(Platform-as-a-Service)に向けた設計を進めた. また, mdxに関する講演活動を国内外において行った\cite{suzumura-axies2021,suzumura-nanotec2021, suzumura-nci2021}. mdxの論文においては、国際会議IEEE IC2E2022(10th IEEE International Conference on Cloud Engineering) に2022年3月末に投稿した. 
% %\cite{suzumura-mdx2022}においても論文を公開した。

