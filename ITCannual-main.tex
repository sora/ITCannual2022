\documentclass[11pt]{jarticle}
\usepackage{ITCannual}
\usepackage{amsmath}
\usepackage{amssymb}
\usepackage{times}
\usepackage{graphicx}

%\usepackage[style=numeric]{biblatex}

\title{データ科学研究部門 研究報告}
\author{小林博樹, 鈴村豊太郎, 空閑洋平, Parajuli Laxmi Kumar, 河村光晶, 姜仁河, 川瀬純也, \\
\textbf{華井雅俊, Li Zihui, 金刺宏樹, 石川正俊, 早川智彦, 黄守仁, 末石智大, 宮下令央, 田畑智志}}

\begin{document}
\maketitle

\section{データ科学研究部門 概要}
データ科学研究部門では、2021年度、教授4名(特任教授1名)、准教授3名(特任准教授1名)、講師3名(特任講師3名)、助教 4 名(特任助教2名)が在籍した。同部門のメンバーは専任教員と特任教員の2つのグループから成る。専任教員はそれぞれが独立して研究活動を行うグループで、特任教員は石川特任教授を中心とする石川グループ研究室である。


% \subsection{専任教員グループの研究テーマ}
%
% \begin{quote}
% \begin{itemize}
% \item 計算機を介した人と生態系のインタラクションの研究(小林)
% \item 大規模グラフニューラルネットワークと様々な実社会問題への応用(鈴村)
% \item データセンタハードウェアへのソフトウェア脆弱試験の適応(空閑)
% \item Title (Kumar)
% \item Title (河村)
% \item データ駆動型知能に基づくアーバンコンピューティング(姜)
% \item 野生動物ワイヤレスセンサネットワーク実証実験基盤構築に向けた研究(川瀬)
% \item グラフニューラルネットワークとその物性予測問題への応用に関する研究(華井)
% \item Title (Zihui)
% \end{itemize}
% \end{quote}

%\subsection{石川研究室全体の研究活動概要}
\subsection{研究報告(石川グループ研究室)}

%  センサやロボットはもちろんのこと、社会・心理現象等も含めて、現実の物理世界は、原則的に並列かつリアルタイムの演算構造を有している。その構造と同等の構造を工学的に実現することは、現実世界の理解を促すばかりでなく、応用上の様々な利点をもたらし、従来のシステムをはるかに凌駕する性能を生み出すことができ、結果として、まったく新しい情報システムを構築することが可能となる。本研究室では、特にセンサ情報処理における並列処理と高速・リアルタイム性を高度に示現する研究として、以下4つの分野での研究を行っている。また、新規産業分野開拓にも力を注ぎ、研究成果の技術移転,共同研究,事業化等を様々な形で積極的に推進している。

% 五感の工学的再構成を目指したセンサフュージョンの研究では、理論並びにシステムアーキテクチャの構築とその高速知能ロボットの開発、その応用としての新規タスクの実現、特に、視・触覚センサによるセンサ情報に基づく人間機械協調システムの開発を行っている。

% ダイナミックビジョンシステムの研究では、高速ビジョンや動的光学系に基づき運動対象の情報を適応的に取得する基礎技術の開発、特に、高速光軸制御や可変形状光学系の技術開発やトラッキング撮像に関する応用システムの開発を行っている。

% 高速三次元形状計測や高速質感計測など、並列処理に基づく高速画像処理技術 (理論、アルゴリズム、デバイス) 開発とその応用システムの実現を目指すシステムビジョンデザインの研究では、特に高速画像処理システムの開発、高速性を利用した新しい価値を創造する応用システムの開発を行っている。

% 実世界における新たな知覚補助技術並びにそれに基づく新しい対話の形の創出を目指すアクティブパーセプション技術の構築とその応用に関する研究では、特に各種高速化技術を用いた能動計測や能動認識を利用した革新的情報環境・ヒューマンインタフェイスの開発を行っている。



\section{データ科学研究部門 教員研究活動}

\subsection{計算機を介した人と生態系のインタラクションの研究(小林 博樹)}

% 本研究室は計算機を介した人と生態系のインタラクションの研究の行っている。これまで人間を対象とした知能情報学の見地を、多様で複雑な実世界の生物・環境・地理学・獣医学領域へ応用・発展させる研究である。研究内容はコンピュータ科学、環境学、メディアアート、など多岐に亘っており、特に、計算機を介した人と生態系のインタラクションHCBI(Human-Computer-Biosphere Interaction)の概念を情報学分野で発表し、このテーマを中心に、環境問題の解決を目的として、国内外で研究活動を独自に行ってきた。古典的なコンピュータ科学では、HCI(Human-Computer Interaction)が主要な研究領域の1つとなるが、本研究室はこの研究領域を地球環境にまで拡大すべく、人間と生態系の調和あるインタラクションを実現するシステムを提案し「時空間スケールの大きい環境問題を自律的に解決する情報基盤技術」として、そのフィールドでの実証実験を試みている。つまり、コンピュータ科学の分野では人間が活動する地理空間を対象とした研究が中心であったが、本研究室は人間が活動していない、情報通信技術の応用が困難な地理空間を対象にした情報デザインと野生動物IoTの研究を行っている。このように本研究室は、情報工学をベースとして、特に計算機を用いて生態系と人間のインタラクションを専門として実績をあげている。2021年度はドコモ・モバイル・サイエンス賞「社会科学部門」(優秀賞)等を受けた。
 
% \begin{受賞}{1}
% \bibitem{kobayashi1-1}
% 小林博樹:情報通信技術の導入が困難なインフラ圏外空間を対象とした情報デザインとIoTの研究,ドコモ・モバイル・サイエンス賞 社会科学部門 優秀賞,2021/9.

% \bibitem{kobayashi1-2}
% Hill Hiroki Kobayashi, Radioactive Live Soundscape, Winner, Universal Design Expert, Institute for Universal Design KG, Germany, 2021/05.
% \end{受賞}

\begin{雑誌論文}{1}
\bibitem{kobayashi1-1}
Wenjing Li, Haoran Zhang, Ryosuke Shibasaki,  Jinyu Chen and Hill Hiroki Kobayashi,  "Mining individual significant places from historical trajectory data", Handbook of Mobility Data Mining, Jan, 2023.

\bibitem{kobayashi1-2}
Wenjing Li, Haoran Zhang, Ryosuke Shibasaki,  Jinyu Chen and Hill Hiroki Kobayashi,  "Mobility pattern clustering with big human mobility data", Handbook of Mobility Data Mining, Jan, 2023.

\bibitem{kobayashi1-3}
Wenjing Li, Haoran Zhang, Jinyu Chen, Peiran Li, Yuhao Yao, Xiaodan Shi,  Mariko Shibasaki, Hill Hiroki Kobayashi, Xuan Song and Ryosuke Shibasaki,  "Metagraph-Based Life Pattern Clustering With Big Human Mobility Data", IEEE Transactions on Big Data, Feb, 2023.

\end{雑誌論文}

\begin{査読付}{1}
\bibitem{kobayashi2-1}
Daisuk\'e Shimotoku, Tian Yuan, Laxmi Kumar Parajuli and Hill Hiroki Kobayashi, "Participatory Sensing Platform Concept for Wildlife Animals in the Himalaya Region, Nepal", Proceedings of 2022 International Conference on Human-Computer Interaction (HCII 2022), 2022.  

\bibitem{kobayashi2-2}
Qaiser Anwar, Muhammad Hanif, Daisuk\'e Shimotoku and  Hiroki Kobayashi, "Driver awareness collision/proximity detection system for heavy vehicles based on deep neural network", International Symposium on Intelligent Unmanned Systems and Artificial Intelligence (SIUSAI 2022) , 2022.  

\bibitem{kobayashi2-3}
Usman Haider, Muhammad Hanif, Laxmi Kumar Parajuli, Hill Hiroki Kobayashi, Daisuk\'e Shimotoku, Ahmar Rashid and Sonia Safeer, "Bioacoustics signal classification using hybrid feature space with machine learning",  International Conference on Computer and Automation Engineering (ICCAE 2023), 2023.  

\end{査読付}

\begin{招待講演}{1}
\bibitem{kobayashi3-1}
Hill Hiroki Kobayashi, mdx: A Cloud Platform for Supporting Data Science and Cross-Disciplinary Research Collaborations, the Nepal JSPS Alumni Association (NJAA), hosted its 7th Symposium, 29 November, 2022.

\bibitem{kobayashi3-2}
小林博樹, mdx を用いたフィールドデータの蓄積と公開の可能性, 東京大学大学院農学生命科学研究科附属演習林 生態水文学研究所100周年記念式典, 8 Dec, 2022.

\end{招待講演}

% %半ページから1ページが文量

\subsection{大規模グラフニューラルネットワークの理論と応用(鈴村 豊太郎)}

本節では2022年度の鈴村豊太郎の研究活動について報告する。 鈴村は、 グラフ構造に対するニューラルネットワークを用いた表現学習 Graph Neural Network (以下、GNNと呼ぶ)の基礎研究及び応用研究に取り組んでいる。 グラフ構造は、 ノードと、 ノード同士を接続するエッジから構成されるデータ構造である。 インターネット上における社会ネットワーク、 購買行動、 サプライチェーン、 金融における決済データ、 交通ネットワーク、 蛋白質相互作用・神経活動・DNAシーケンス配列内の依存性、 物質の分子構造、 人間の骨格ネットワーク、 概念の関係性を表現した知識グラフなど、 グラフ構造として表現できる応用先は枚挙に暇がない。
\par
当該研究領域における研究として、時系列・動的に変化する大規模グラフに対するGNNモデルの研究を行った。
動的グラフに対応するGNNモデルはすでに数多く提案されているが、いずれも短期的なデータの変化しか考慮されておらず、実世界で扱われている長期的なグラフデータでは長期的なコンテキストを捉えることができない問題が潜在的に存在していた。この問題に対して、時間幅が非常に長いグラフデータの性質も捉えることができる Spectral Waveletを提案した(AAAI 2023 \cite{aaai-deft}, TLMR \cite{feta})。
%知識グラフに関する研究成果はWWW 2023 \cite{kg-kp}
% 当該研究領域における研究として、時系列・動的に変化する大規模グラフに対するGNNモデルの研究を行った。実世界では時間幅が非常に長いデータを扱うこともあるが既存の動的グラフへのGNNの研究ではそのような点を考慮していない。この問題に対して、学習可能な Spectral Waveletを提案し、AAAI 2023 \cite{aaai-deft}, WWW 2023 \cite{deft}、 TLMR \cite{feta}に採択された。

GNNに関する応用研究も進めている。金融領域においては不正検出に関するGNNモデルの検証を行い、取引ネットワークをヘテロジニアスなグラフ構造に拡張することによりモデル性能の向上を達成した \cite{eth-gnn}。マテリアルズ・インフォマティクスの分野においては、情報基盤センターの芝隼人先生とガラス物質の形成過程モデルに対して高精度なGNNモデルを提案し\cite{botan}、また当該分野における本質的な問題に対する手法として、インバランスなデータの問題を解消するための手法 \cite{xsig-limin}および外挿のためのモデル構築を行った\cite{xsig-takashige}。E-Commerceの領域においては知識グラフを用いた商品推薦手法を提案した (SIGIR 2023 \cite{sigir})。

また、理論モデルの実世界への検証と応用サイドから意味のある研究テーマを発掘するため、企業との共同研究とも進めている。まず、自動車の走行軌跡データから次の位置や経路を予測し、ロケーションリコメンデーションなどに応用するための手法をトヨタ自動車と探求している。走行軌跡データは緯度・経度及び時刻のシーケンスデータとなるが、それを用いると運転行動パターンを捉えることができる。
% まずシーケンスデータからグラフ構造を構築し、そのグラフ構造からGraphormerというニューラルネットワークモデルを走行軌跡データのパターンを捉えられるようなニューラルネットワークのモデルを提案した。
この行動パターンを捉えるために、シーケンスデータをグラフ構造として表現し、Graphormerをベースにした新たなモデルを提案し、他の既存手法よりも高い精度でパターンを予測することを確認した (ECML-PKDD \cite{stgtrans} 査読中)。
% この新たなモデルを他の既存手法と比較し、より高い精度で走行パターンを予測する事を確認した。本研究の成果を ECML-PKDD \cite{stgtrans}に提出した。
来年度はモビリティにおける様々な領域に応用できるように、走行軌跡データや実世界の地図データなどから事前学習モデルを構築する予定である。また、その他に都市全体の二酸化炭素排出量を抑制するために交通流を分散するための手法をこれらの事前学習モデルと深層強化学習を用いて設計・実装する予定である。

また、エス・エム・エス社との共同研究では、介護や医療領域における人材紹介の推薦システムに関する研究を行った。超高齢化社会に突入する中、介護や医療領域における人材不足は深刻であり、より精度の高い人材マッチングが不可欠である。この問題に対して、深層強化学習を用いた人材マッチング数の最適化手法を提案し、特定の求職者・事業者側に偏ってしまう従来の推薦・マッチング手法に対して、偏りを解消できることを確認した (人工知能学会\cite{sms} 6月発表予定)。来年度に関しては更に実データでの検証を進め、企業側での要望を取り入れ、実ビジネスが持つ制約条件を取り入れた最適化モデルを提案していく予定である。また、モデルにおいて求職者と求人側での動的な二部グラフの関係性及び知識グラフを用いてより精度高いモデルを構築していく予定である。

 日本経済新聞社(以下、日経)との共同研究も2022年10月から開始している。日経ではニュースサイトにおける記事に対してより高度な機械学習による推薦システムを目指しており、2022年度は日経側での問題設定やデータの理解を図り、研究テーマの設定に主に取り組んだ。2023年度は推薦問題や広告配信など様々な領域に応用できるように、ユーザーの行動モデルを統一的に表現する事前学習モデルを構築するべく、自己教師付き学習 (Self-Supervised Learning)を用いた手法を設計・実装する予定である。

 また、GNNの概要と最新研究動向に関する記事を人工知能学会誌\cite{jsai-gnn}に寄稿し、Federated Learning(連合学習)の英語書籍向けに Federated Learningを用いた金融不正検知に関する手法を執筆した\cite{fl-book}。mdxプロジェクトに関する第一弾の国際学会論文として IEEE CBDCom\cite{mdx}にて論文発表を行った。雑誌、査読付き論文、招待講演等のリストは以下の通りである。


% How Expressive are Transformers in Spectral Domain for Graphs? \cite{feta}
% Learnable Spectral Wavelets on Dynamic Graphs to Capture Global Interactions \cite{deft}
% Can Persistent Homology provide an efficient alternative for Evaluation of Knowledge Graph Completion Methods? \cite{kg-kp}


% Spatio-Temporal Meta-Graph Learning for Traffic Forecasting \cite{megacrn}

% Ethereum Fraud Detection with Heterogeneous Graph Neural Networks \cite{eth-gnn}

% Federated Learning for Collaborative Financial Crimes Detection \cite{fl-book}



%  本節では2021 年度の鈴村豊太郎の研究活動について報告する。 2021年4月に本学に着任し、 グラフ構造に関するニューラルネットワークを用いた表現学習 Graph Neural Network (以下、GNNと呼ぶ)の基礎研究及びその様々な応用研究に取り組んでいる。 グラフ構造は、 ノードと、 ノード同士を接続するエッジから構成されるデータ構造である。 インターネット上における社会ネットワーク、 購買行動、 サプライチェーン、 金融における決済データ、 交通ネットワーク、 蛋白質相互作用・神経活動・DNAシーケンス配列内の依存性、 物質の分子構造、 人間の骨格ネットワーク、 概念の関係性を表現した知識グラフなど、 グラフ構造として表現できる応用先は枚挙に暇がない。
% \par
% 当該研究領域において、時系列・動的に変化する大規模グラフに対するGNNモデルの研究を行った。分散計算環境においてスケールするGNNモデルを提唱し、その成果は高性能計算分野におけるトップカンファレンスSC2021\cite{suzumura-sc2021}に採択された。 また、金融領域における不正検知手法として、TransformerアーキテクチャをベースにしたGNN手法を提案し、国際会議 IEEE SMDS 2021\cite{suzumura-smds21}に採択された。 また、 GNNに関する招待講演\cite{suzumura-canon2021}を行った。

% これらの研究に続いて、推薦システムへのGNNモデルに関する研究を開始している。実データ・実問題に基づいた、社会実装を見据えた研究を進めるべく、医療・介護領域における人材推薦としてエス・エム・エス社、自動車における経路推薦としてトヨタ社と共同研究を2023年4月から本格的に開始する。また、国立研究開発法人物質・材料研究機構NIMSが主導する「マテリアル先端リサーチインフラ」プロジェクトの本学拠点の一貫で、材料情報科学 Materials Informaticsへの研究も開始している。
%  データ科学・データ利活用のためのクラウド基盤 mdx プロジェクトにおいて、 今年度は 課金付き運用開始に向けたシステム拡張、スポットVM、データ共有機構(Platform-as-a-Service)に向けた設計を進めた。 また、 mdxに関する講演活動を国内外において行った\cite{suzumura-axies2021、suzumura-nanotec2021、 suzumura-nci2021}。 mdxの論文においては、国際会議IEEE IC2E2022(10th IEEE International Conference on Cloud Engineering) に2022年3月末に投稿した。 
% %\cite{suzumura-mdx2022}においても論文を公開した。




% 
% bibitem を作る

\begin{査読付}{9}
\bibitem{feta}
Anson Bastos, Abhishek Nadgeri, Kuldeep Singh, Hiroki Kanezashi, Toyotaro Suzumura, Isaiah Onando Mulang', 
"How Expressive are Transformers in Spectral Domain for Graphs?"
Transactions on Machine Learning Research, 2022, https://openreview.net/forum?id=aRsLetumx1

\bibitem{deft}
Anson Bastos, Abhishek Nadgeri, Kuldeep Singh, Toyotaro Suzumura, Manish Singh,
"Learnable Spectral Wavelets on Dynamic Graphs to Capture Global Interactions"
The 37th AAAI Conference on Artificial Intelligence (AAAI 2023), https://arxiv.org/abs/2211.11979

\bibitem{kg-kp}
Anson Bastos, Kuldeep Singh, Abhishek Nadgeri, Johannes Hoffart, Toyotaro Suzumura, Manish Singh,
"Can Persistent Homology provide an efficient alternative for Evaluation of Knowledge Graph Completion Methods?"
In proceedings of The Web Conference 2023 (WWW'23)


\bibitem{gqsm}
Chinthaka Weerakkody, Miyuru Dayarathna, Sanath Jayasena, Toyotaro Suzumura
"Guaranteeing Service Level Agreements for Triangle Counting via Observation-based Admission Control Algorithm"
2022 IEEE 15th International Conference on Cloud Computing (CLOUD)

\bibitem{mdx}
Toyotaro Suzumura, Akiyoshi Sugiki, Hiroyuki Takizawa, Akira Imakura, Hiroshi Nakamura, Kenjiro Taura, Tomohiro Kudoh, Toshihiro Hanawa, Yuji Sekiya, Hiroki Kobayashi, Shin Matsushima, Yohei Kuga, Ryo Nakamura, Renhe Jiang, Junya Kawase, Masatoshi Hanai, Hiroshi Miyazaki, Tsutomu Ishizaki, Daisuke Shimotoku, Daisuke Miyamoto, Kento Aida, Atsuko Takefusa, Takashi Kurimoto, Koji Sasayama, Naoya Kitagawa, Ikki Fujiwara, Yusuke Tanimura, Takayuki Aoki, Toshio Endo, Satoshi Ohshima, Keiichiro Fukazawa, Susumu Date, Toshihiro Uchibayashi
"mdx: A Cloud Platform for Supporting Data Science and Cross-Disciplinary Research Collaborations", 2022 IEEE Intl Conf on Dependable, Autonomic and Secure Computing, Intl Conf on Pervasive Intelligence and Computing, Intl Conf on Cloud and Big Data Computing, Intl Conf on Cyber Science and Technology Congress (DASC/PiCom/CBDCom/CyberSciTech)


\bibitem{megacrn}
Renhe Jiang, Zhaonan Wang, Jiawei Yong, Puneet Jeph, Quanjun Chen, Yasumasa Kobayashi, Xuan Song, Shintaro Fukushima, Toyotaro Suzumura
"Spatio-Temporal Meta-Graph Learning for Traffic Forecasting"
37th AAAI Conference on Artificial Intelligence (AAAI 2023), https://arxiv.org/abs/2211.14701

\bibitem{botan}
Shiba, Hayato, Masatoshi Hanai, Toyotaro Suzumura, and Takashi Shimokawabe. "BOTAN: BOnd TArgeting Network for prediction of slow glassy dynamics by machine learning relative motion." The Journal of Chemical Physics 158, no. 8 (2023): 084503.

\bibitem{eth-gnn}
Hiroki Kanezashi, Toyotaro Suzumura, Xin Liu, Takahiro Hirofuchi,
"Ethereum Fraud Detection with Heterogeneous Graph Neural Networks"
28TH ACM SIGKDD Conference on Knowledge Discovery and Data Mining (KDD'22) Workshop on Mining and Learning with Graphs

\bibitem{sms}
脇 聡志, 鈴村 豊太郎, 金刺 宏樹, 小林 秀, "強化学習によるマッチング数を最大化するジョブ推薦システム." 第37回人工知能学会全国大会 (2023),  一般社団法人 人工知能学会, 2023.


% \bibitem{suzumura-sc2021}
% Venkatesan T. Chakaravarthy, Shivmaran S. Pandian, Saurabh Raje, Yogish Sabharwal, Toyotaro Suzumura, Shashanka Ubaru, 
% "Efficient scaling of dynamic graph neural networks". SC2021(The International Conference for High Performance Computing, Networking, Storage, and Analysis)

% \bibitem{suzumura-smds21}
% Shilei Zhang, Toyotaro Suzumura, Li Zhang, "DynGraphTrans: Dynamic Graph Embedding via Modified Universal Transformer Networks for Financial Transaction Data", IEEE SMDS 2021 (International Conference on Smart Data Services) 
\end{査読付}

\begin{著書}{2}

\bibitem{fl-book}
Toyotaro Suzumura, Yi Zhou, Ryo Kawahara, Nathalie Baracaldo, Heiko Ludwig,
"Federated Learning for Collaborative Financial Crimes Detection".
Ludwig, H., Baracaldo, N. (eds) Federated Learning. Springer, Cham. https://doi.org/10.1007/978-3-030-96896-0\_20

\bibitem{jsai-gnn}
鈴村 豊太郎, 金刺 宏樹, 華井 雅俊, グラフニューラルネットワークの広がる活用分野, 人工知能, 2023, 38 巻, 2 号, p. 139-148, 公開日 2023/03/02, Online ISSN 2435-8614, Print ISSN 2188-2266, https://doi.org/10.11517/jjsai.38.2\_139

\end{著書}

% \begin{発表}{1}
% \bibitem{suzumura-axies2021}
% 鈴村豊太郎, "データ活用社会創成プラットフォームmdxの設計・実装・運用〜多様な学際領域における共創に向けて~", 大学ICT推進協議会2021年度年次大会(AXIES2021), 2021年12月
% \end{発表}

% \begin{招待講演}{1}

% %\bibitem{suzumura-mdx2022}
% %Toyotaro Suzumura, Akiyoshi Sugiki, Hiroyuki Takizawa, Akira Imakura, Hiroshi Nakamura, Kenjiro Taura, Tomohiro Kudoh, Toshihiro Hanawa, Yuji Sekiya, Hiroki Kobayashi, Shin Matsushima, Yohei Kuga, Ryo Nakamura, Renhe Jiang, Junya Kawase, Masatoshi Hanai, Hiroshi Miyazaki, Tsutomu Ishizaki, Daisuke Shimotoku, Daisuke Miyamoto, Kento Aida, Atsuko Takefusa, Takashi Kurimoto, Koji Sasayama, Naoya Kitagawa, Ikki Fujiwara, Yusuke Tanimura, Takayuki Aoki, Toshio Endo, Satoshi Ohshima, Keiichiro Fukazawa, Susumu Date, Toshihiro Uchibayashi, "mdx: A Cloud Platform for Supporting Data Science and Cross-Disciplinary Research Collaborations", https://arxiv.org/abs/2203.14188

% \bibitem{suzumura-nci2021}
% 鈴村豊太郎, “mdx: A platform for the data-driven future”、オーストラリア国立研究所NCI(National Computational Infrastructure)-Fujitsu HPC, Cloud and Data Futures Workshop, 2022年02月

% \bibitem{suzumura-nanotec2021}
% 鈴村豊太郎, "データ活用社会創成プラットフォームmdxにおけるマテリアルズ・インフォマティクス研究・共創に向けて", 第20回ナノテクノロジー総合シンポジウム, 2022年01月

% \bibitem{suzumura-canon2021}
% 鈴村豊太郎, "人工知能を支えるグラフニューラルネットワークの最新動向", 2021年度キャノングローバル戦略研究所主催「経済・社会との分野横断的研究会」, 2021年12月


% \end{招待講演}




\subsection{データセンタハードウェアへのソフトウェア脆弱試験の適応(空閑 洋平)}

現在のデータセンタ環境では、機械学習やニューラルネットワークの学習,推論を高速化する専用アクセラレータが広く使用されるようになった。専用アクセラレータを用いた計算環境は、既存のCPUを中心に構成されていたソフトウェア環境に比べて、プロセッサやデバイスドライバ、デバイス間通信が専用に設計され、CPUをバイパスしてデバイス間で直接データ通信されるため、デバイスのデータ通信の把握や可視化が困難なブラックボックス化が進んでいる。今後、専用アクセラレータを中心とした次世代のデータセンタ環境では、CPUをバイパスするデバイス間通信が増加することで、セキュリティ監視や脆弱性試験、管理手法、データ通信内容の可視化手法といった、普段CPU環境で実施している運用課題が顕在化すると考えられる。

今年度は、CPUからデータセンタハードウェアを直接操作するために設計した独自Remote DMA機能を用いた研究を実施した\cite{ykuga39987672,ykuga41835070,ykuga37056192}。
特に、今年度はソフトウェアでプログラム可能なメモリデバイスを提案し、ソフトウェアでアクセラレータやストレージのデータ通信内容の観測と書き換えが可能なことを確認できたため、来年度はアクセラレータに対する脆弱性試験を予定している。
また、昨年から引き続き、CPUをバイパスするNIC型ネットワークルータアーキテクチャの検討を実施し、今年度は経路表のみをCPUで処理するハイブリットアプローチを提案している\cite{ykuga36919054}。
mdxの機能高度化に関する研究については、データ転送機能の高性能化手法の検討と、mdx上でのkubernetes基盤構築に関する機能開発について招待論文で報告した\cite{ykuga41835081,ykuga41534619}。
その他の活動としては、東京大学のZoomデータを用いて、広域ネットワーク品質解析の手法を提案した\cite{ykuga40356877}。

\begin{発表}{3}
% \bibitem{ykuga41431156}
% Toyotaro Suzumura, Akiyoshi Sugiki, Hiroyuki Takizawa, Akira Imakura, Hiroshi Nakamura, Kenjiro Taura, Tomohiro Kudoh, Toshihiro Hanawa, Yuji Sekiya, Hill Hiroki Kobayashi, Shin Matsushima, Yohei Kuga, Ryo Nakamura, Renhe Jiang, Junya Kawase, Masatoshi Hanai, Hiroshi Miyazaki, Tsutomu Ishizaki, Daisuké Shimotoku, Daisuke Miyamoto, Kento Aida, Atsuko Takefusa, Takashi Kurimoto, Koji Sasayama, Naoya Kitagawa, Ikki Fujiwara, Yusuke Tanimura, Takayuki Aoki, Toshio Endo, Satoshi Ohshima, Keiichiro Fukazawa, Susumu Date, Toshihiro Uchibayashi, mdx: A Cloud Platform for Supporting Data Science and Cross-Disciplinary Research Collaborations., CoRR, abs/2203.14188, May, 2022.

\bibitem{ykuga41835081}
中村遼, 空閑洋平, 複数コネクションを用いる高速なscpの実装, 研究報告システムソフトウェアとオペレーティング・システム(OS), Feb, 2023.

\bibitem{ykuga41835070}
空閑洋平, 中村遼, ソフトウェアメモリを用いたNVMeコマンドのキャプチャ, 研究報告システムソフトウェアとオペレーティング・システム(OS), Feb, 2023.

\end{発表}

\begin{査読付}{3}
\bibitem{ykuga40356877}
空閑洋平, 中村遼, 遠隔会議システムの計測データを用いた広域ネットワーク品質計測, インターネットと運用技術シンポジウム論文集, 2022, Dec, 2022.

\bibitem{ykuga39987672}
Shu Anzai, Masanori Misono, Ryo Nakamura, Yohei Kuga, Takahiro Shinagawa, Towards isolated execution at the machine level, Proceedings of the 13th ACM SIGOPS Asia-Pacific Workshop on Systems, 23 Aug, 2022.

\bibitem{ykuga36919054}
Yukito Ueno, Ryo Nakamura, Yohei Kuga, Hiroshi Esaki, Pktpit: separating routing and packet transfer for fast and scalable software routers, Proceedings of the 37th ACM/SIGAPP Symposium on Applied Computing, 25 Apr, 2022.

\end{査読付}

\begin{招待講演}{1}
\bibitem{ykuga37056192}
空閑洋平, クラウド環境におけるネットワークインタフェースの高機能化, 第155回 システムソフトウェアとオペレーティング・システム研究会, 26 May, 2022.

\end{招待講演}

\begin{招待論文}{1}
\bibitem{ykuga41534619}
杉木章義, 空閑洋平, 竹房あつ子, 藤原一毅, 合田憲人, 中村遼, 塙敏博, 鈴村豊太郎, 宮本大輔, 田浦健次朗, 伊達進, 建部修見, データ活用社会創成に向けた基盤ソフトウェア環境の構築, 学術情報処理研究, 26, 1, pp1-9, Dec, 2022.

\end{招待論文}

\subsection{Research on the use of ICT for biodiversity conservation(Parajuli Laxmi Kumar)}

I have been working with Professor Kobayashi on the use of space technology for biodiversity conservation in Nepal. During fiscal year 2022, I discussed with the research officers in National Trust for Nature Conservation (NTNC), Nepal about the possibility of incorporating high technology for nature conservation in Nepal. We also discussed with the locals in Nepal about their needs and how technological intervention is necessary to address their problems. A technical cooperation request was submitted by the NTNC on the use of space technology such as GPS radio collars, drones, infrared sensors, bioacoustics, camera traps and LiDAR technology. I was also involved in writing a perspective paper (Uesaka et al., 2023) in collaboration with the researchers in the laboratory of Professor Kobayashi and two scientists in NTNC, Nepal. 
 

\begin{雑誌論文}{1}

\bibitem{Uesaka2023HCII}
Uesaka L, Khatiwada AP, Shimotoku D, Parajuli LK, Pandey MR, Kobayashi HH (2023). Applications of Bioacoustics Human Interface System for Wildlife Conservation in Nepal. Accepted for publication in Human Computer Interaction (HCI) International 2023.

\end{雑誌論文}

\subsection{Title(河村 光晶)}
\begin{雑誌論文}{3}

\bibitem{osada.pccp.24.21705}
Wataru Osada, Shunsuke Tanaka, Kozo Mukai, Mitsuaki Kawamura, YoungHyun Choi, Fumihiko Ozaki, Taisuke Ozaki and Jun Yoshinobu, 
"Elucidation of the atomic-scale processes of dissociative adsorption and spillover of hydrogen on the single atom alloy catalyst Pd/Cu(111)",
Physical Chemistry and Chemical Physics 24, 36, 21705-21713, 2022.

\bibitem{hirai.jacs.144.17857}
Daigorou Hirai, Keita Kojima, Naoyuki Katayama, Mitsuaki Kawamura, Daisuke Nishio-Hamane, and Zenji Hiroi,
"Linear Trimer Molecule Formation by Three-Center-Four-Electron Bonding in a Crystalline Solid RuP",
Journal of the American Chemical Society 144, 39, 17857-17864, 2022.

\bibitem{koshoji.prm.6.114802}
Ryotaro Koshoji, Masahiro Fukuda, Mitsuaki Kawamura, Taisuke Ozaki,
"Prediction of quaternary hydrides based on densest ternary sphere packings",
Physical Review Materials 6, 11, 114802.1-9, 2022.

\end{雑誌論文}



\subsection{データ駆動型知能に基づくアーバンコンピューティング(姜 仁河)}
%  本節では2021年度の姜仁河の研究活動について報告する。近年、都市のスマート管理、スマートシティは新しい科学技術分野として各国の学術界、産業界および各国政府から非常に重視されている。モバイルデータ、IoTセンサデータ、衛星画像、交通プローブデータ、災害データなどダイナミックなリアルタイム時空間ビッグデータが入手可能な環境が急速に整いつつあり、健康や医療サービスデータ、購買履歴データ、物流・商流などの経済データも積極的に活用されている。これらのデータを統合した形で人々や企業の活動、交通・物流・商流から都市の拡大・環境変化、社会経済システムの変質・変動までを包含するデジタル社会空間のあらゆる課題を解決する。これを目的にして、引き続き2021年度、私は時空間データインテリジェンスについて研究活動を行ってきた。

% 時空間データを一定の時間間隔・空間単位で集計すると、汎用的に3Dテンソル「T×N×C」(Tはタイムステップの数、Nは交差点やリンクやメッシュグリッドや不規則な区域の数、Cは情報量の数)で表現できる。時間軸Tから見ると、将来の予測値は最近の観測値と過去に現れた定期的なパターンに依存する。空間軸Nから見ると、ある交差点や区域内の交通量は近隣や遠くにあるものに影響される。このような3Dテンソル「T×N×C」に含まれる超複雑な時空間依存関係をモデリングするために、一連の深層学習モデル(Temporal Convolution Network、Recurrent Neural Network 、Graph Convolution Network、Attention Mechanism)を設計・開発し、群衆密度、タクシー・シェアサイクル需要、救急車需要、新型コロナ感染者数を高精度で予測可能にした。関連成果はデータサイエンス分野のトップカンファレンスKDD2021、ICDE 2021、ECMLPKDD2021、CIKM2021、AAAI2022\cite{JIANG2-1,JIANG2-2,JIANG2-3,JIANG2-4,JIANG2-5,JIANG2-6}及びトップジャーナルACM TIST 2022、ACM TKDD 2021、IEEE TKDE2021、IEEE TVCG2021\cite{JIANG1-1,JIANG1-2,JIANG1-3,JIANG1-4,JIANG1-5}にて発表された。特に、交通流モデリング・予測の深層学習ベンチマークDL-Traff\cite{JIANG2-3}は、国際トップカンファレンスCIKM 2021において最優秀リソース論文賞(Best Resource Paper Runner-Up Award)を受賞した。また、Yahoo Japan Researchとの共同研究によって、オープンデータサイエンスの促進に努めた。個人情報・プライバシー問題に配慮したうえで、メッシュベースの東京都・大阪市の群衆密度・移動データをデータサイエンス分野トップジャーナルであるIEEE TKDE2021\cite{JIANG1-4}において公開した。


\begin{雑誌論文}{1}
\bibitem{JIANG1-1}
Renhe Jiang, Zekun Cai, Zhaonan Wang, Chuang Yang, Zipei Fan, Quanjun Chen, Xuan Song, and Ryosuke Shibasaki, "Predicting Citywide Crowd Dynamics at Big Events: A Deep Learning System", ACM Trans. Intell. Syst. Technol. (TIST), 13, 2, Article 21, April 2022.
\bibitem{JIANG1-2}
Zipei Fan, Chuang Yang, Zhiwen Zhang, Xuan Song, Yinghao Liu, Renhe Jiang, Quanjun Chen, and Ryosuke Shibasaki, "Human Mobility-based Individual-level Epidemic Simulation Platform", ACM Trans. Spatial Algorithms Syst. (TSAS), 8, 3, Article 19, September 2022.
\bibitem{JIANG1-3}
Chuang Yang, Zhiwen Zhang, Zipei Fan, Renhe Jiang, Quanjun Chen, Xuan Song, Ryosuke Shibasaki, "EpiMob: Interactive Visual Analytics of Citywide Human Mobility Restrictions for Epidemic Control", IEEE Transactions on Visualization and Computer Graphics (TVCG), 2022.
\bibitem{JIANG1-4}
Yudong Tao, Chuang Yang, Tianyi Wang, Erik Coltey, Yanxiu Jin, Yinghao Liu, Renhe Jiang, Zipei Fan, Xuan Song, Ryosuke Shibasaki, Shu-Ching Chen, Mei-Ling Shyu, Steven Luis, "A Survey on Data-Driven COVID-19 and Future Pandemic Management", ACM Computing Surveys (CSUR), 2022.
\bibitem{JIANG1-5}
Yongkang Li, Zipei Fan, Du Yin, Renhe Jiang, Jinliang Deng, Xuan Song, "HMGCL: Heterogeneous Multigraph Contrastive Learning for LBSN Friend Recommendation", World Wide Web (WWW), 2022.
\bibitem{JIANG1-6}
Renhe Jiang, Quanjun Chen, Zekun Cai, Zipei Fan, Xuan Song, Kota Tsubouchi, and Ryosuke Shibasaki, “Will You Go Where You Search? A Deep Learning Framework for Estimating User Search-and-Go Behavior”, Neurocomputing, Volume 472, 2022, Pages 338-348.
\bibitem{JIANG1-7}
Jinliang Deng, Xiusi Chen, Renhe Jiang, Xuan Song, Ivor W. Tsang, "A Multi-view Multi-task Learning Framework for Multi-variate Time Series Forecasting", IEEE Transactions on Knowledge and Data Engineering (TKDE), 2022.

\end{雑誌論文}

\begin{査読付}{1}
\bibitem{JIANG2-1}
Renhe Jiang, Zhaonan Wang, Jiawei Yong, Puneet Jeph, Quanjun Chen, Yasumasa Kobayashi,
Xuan Song, Shintaro Fukushima, Toyotaro Suzumura, "Spatio-Temporal Meta-Graph Learning for Traffic Forecasting", Proceedings of Thirty-Seventh AAAI Conference on Artificial Intelligence (AAAI), 2023.
\bibitem{JIANG2-2}
Renhe Jiang, Zekun Cai, Zhaonan Wang, Chuang Yang, Zipei Fan, Quanjun Chen, Kota Tsubouchi, Xuan Song, Ryosuke Shibasaki, "Yahoo! Bousai Crowd Data: A Large-Scale Crowd Density and Flow Dataset in Tokyo and Osaka", Proc. of 2022 IEEE International Conference on Big Data (BigData), 2022.
\bibitem{JIANG2-3}
Xinchen Hao, Renhe Jiang, Jiewen Deng, Xuan Song, "The Impact of COVID-19 on Human Mobility: A Case Study on New York", Proc. of 2022 IEEE International Conference on Big Data (BigData), 2022.
\bibitem{JIANG2-4}
Zheng Dong, Quanjun Chen, Renhe Jiang, Hongjun Wang, Xuan Song, Hao Tian, "Learning Latent Road Correlations from Trajectories", Proc. of 2022 IEEE International Conference on Big Data (BigData), 2022.
\bibitem{JIANG2-5}
Zipei Fan, Xiaojie Yang, Wei Yuan, Renhe Jiang, Quanjun Chen, Xuan Song, Ryosuke Shibasaki, "Online Trajectory Prediction for Metropolitan Scale Mobility Digital Twin", Proc. of 30th ACM SIGSPATIAL International Conference on Advances in Geographic Information Systems (SIGSPATIAL), 2022. 
\bibitem{JIANG2-6}
Haoyuan Ma, Mintao Zhou, Xiaodong Ouyang, Du Yin, Renhe Jiang, Xuan Song, "Forecasting Regional Multimodal Transportation Demand with Graph Neural Networks: An Open Dataset", Proc. of 2022 IEEE International Intelligent Transportation Systems Conference (ITSC), 2022.
\bibitem{JIANG2-7}
Qi Cao, Renhe Jiang, Chuang Yang, Zipei Fan, Xuan Song, Ryosuke Shibasaki, "MepoGNN: Metapopulation Epidemic Forecasting with Graph Neural Networks", Proceedings of the European Conference on Machine Learning and Principles and Practice of Knowledge Discovery in Databases (ECML PKDD), 2022. 
\bibitem{JIANG2-8}
Hangli Ge, Lifeng Lin, Renhe Jiang, Takashi Michikata, Noboru Koshizuka, "Multi-weighted Graphs Learning for Passenger Count Prediction on Railway Network," 2022 IEEE 46th Annual Computers, Software, and Applications Conference (COMPSAC), 2022.
\end{査読付}



\subsection{野生動物ワイヤレスセンサネットワーク実証実験基盤構築に向けた研究(川瀬 純也)}
%  本研究室では、野生動物装着型ワイヤレスセンサーネットワーク(以下:野生動物WSN)機構による自然環境でのデータ収集手法の開発と、それによって得られるデータの解析手法についての研究を行っている。これらを実現するためにはいくつかの課題が存在している。そのひとつは、野生動物に装着可能なサイズ・重量の制限からモバイル端末のバッテリーが限られ、センサーを十分に稼働させることができずにデータ収集に支障をきたす点である。
 
% これを解決するため、野生動物の群れと個体の個性に着目し、野生動物WSNを構築するために必要な機能を群れ内に分散させる機構の開発を行っている。通例の群れを成す動物を対象とした野生動物WSNでは、群れ内の野生動物1個体にモバイル端末を装着し、その1つに機能を集約する。しかし、群れ内の複数の個体に端末を装着し、各々に機能を分散させることで、ひとつひとつのバッテリー消費量を低減させることができる。また、個体の個性に合わせ、活動的な個体には他の群れとの接触を判定しマルチホップ通信を開始する機能を持たせたり、群れの中心であり生存率が高いと考えられる個体にはデータストアの機能を持たせたりするなど、各々の野生動物の特徴に合わせて効率的なデータ収集と回収の実現を目指す。
 
% 現在、モバイル端末の試作品を作成し、今後評価実験等を行う予定である。この研究の一部は『通信スケジュールが不確定な野生動物IoT網における効率的かつ精確なデータ共有手法の開発』と題して、公益財団法人GMOインターネット財団による2021年度GMO研究助成制度の助成を受け行われた。コロナ禍により研究活動が停滞したため、2022年度も引き続き助成を受ける予定である。
 
% 今後は、飼育動物や放牧場などを活用した野生動物WSNの評価実験環境を目指し、将来的にはより自然環境に近い実証実験環境の整備などを視野に入れて研究を進めていく。

\input{Kawase/ITCannual-list-Kawase}

\subsection{グラフニューラルネットワークとその物性予測問題への応用に関する研究(華井 雅俊)}

本節では、2022年度の華井雅俊の研究活動について報告する。グラフニューラルネットワーク(Graph Neural Network, GNN) とその物性予測問題への応用に関する研究に取り組んでいる。
電池、半導体、触媒、医薬品などの材料開発・材料研究の全般において、膨大にある候補材料のさまざまな物性を比較解析することが不可欠であるが、それら候補全てを実際に作り検証することは現実的でない。そのため分子構造などの比較的簡単に得られる物質情報から目的の物性を予測・計算することが重要である。近年では、分子構造(グラフ)データとグラフニューラルネットワークを利用した物性値予測モデルの研究が盛んになってきている。2021年度に引き続き、Stanford Universityが取りまとめるOpen Graph Benchmark (OGB) やCMUとFacebookが主導するOpen Catalyst Project (OCP) などの物性予測問題ベンチマークが機械学習系研究コミュニティで取り上げられ、ますますの盛り上がりを見せている。

2022年度は、GNNを用いた物理問題へのアプローチに関して大きく2の方向性から取り組んでいる。1つは、既存GNNモデルを物理の問題へ応用した際に現れる機械学習手法の限界に関する研究である。機械学習で典型的な、画像処理や自然言語処理では注力されないが物理の問題では非常に重要となる外挿予測とデータの不均衡性に関して特に取り組んだ~\cite{xsig-limin-hanai,xsig-takashige-hanai}。
もう1つは対象の物質により注力した応用研究である。具体的には、ガラスのダイナミクスの予測問題に着手した~\cite{gnn-glass}。 ガラスの振る舞いをグラフを用いてモデル化しGNNを用いることで、分子動力学などのシミュレーション結果を詳細に予測した。
また、その他のGNN応用とも共通の課題として鈴村研究室メンバーとの共同研究も行っており、例えば交通システムの問題に関して研究を行った~\cite{stgtrans-xiaohang}。

また、業務では情報基盤センターが進めるmdxに関して、物性研究や材料開発で得られるデータの利活用を進めている。本年度は物性データに特化したペタスケールストレージをmdxに連携させるシステムを設計し導入を行った。

% 一般に、ある物性値が広範囲な材料群に対し既知である場合予測モデルを構築することが可能となるが、しかし一方で、多くの物性値においては既知である材料が少数であり学習データが不足しているため、実用精度の予測モデルを構築することは難しい。同一の物性であってもパラメータや実験条件が共通化されていないと予測モデルの構築は難しいことが知られ、既存の物性予測の研究では、共通の条件で整理された大規模データが主に利用される(例えば、上のコンペティションなど)。小規模に限定されるデータ、例えば計算コストの膨大なシミュレーション値や実験データ、において、機械学習の利用は限定的であり、大きな研究課題の1つとなっている。

% 我々の研究チームはこのような少規模データに着目し研究を開始した。2021年度下半期は新手法提案への準備としてデータの収集に注力し研究を行った。機械学習分野や材料研究分野で用いられるオープンデータに加え、同学の工学部の研究チームへコンタクトし、スパコンスケールの計算資源を利用し得られた高価なシミュレーション値や実際の実験データに関してヒアリングを行い、データ収集を開始した。
% また、本部門で開発のすすめるmdxにおいては材料系研究への利用促進を行っており、本研究の中間報告として第20回ナノテクノロジー総合シンポジウムにて発表し、IEEE IC2E 2022への投稿論文にて材料系研究におけるクラウド基盤の利活用をまとめた。

% % 2021年度は主に、分野の調査と


% 本節では、2021年度の華井雅俊の研究活動について報告する。2021年9月の本学着任から、グラフニューラルネットワークとその物性予測問題への応用に関する研究に取り組んでいる。

% 電池、半導体、触媒、医薬品などの材料開発・材料研究の全般において、膨大にある候補材料のさまざまな物性を比較解析することが不可欠であるが、それら候補全てを実際に作り検証することは現実的でない。そのため分子構造などの比較的簡単に得られる物質情報から目的の物性を予測・計算することが重要である。近年では、分子構造(グラフ)データとグラフニューラルネットワークを利用した物性値予測モデルの研究が盛んになってきている。特に2021年度はStanford Universityが取りまとめるOpen Graph Benchmark (OGB) やCMUとFacebookが主導するOpen Catalyst Project (OCP) などの機械学習系研究コミュニティのコンペティションで物性予測問題が取り上げられた初めての年であった。

% 一般に、ある物性値が広範囲な材料群に対し既知である場合予測モデルを構築することが可能となるが、しかし一方で、多くの物性値においては既知である材料が少数であり学習データが不足しているため、実用精度の予測モデルを構築することは難しい。同一の物性であってもパラメータや実験条件が共通化されていないと予測モデルの構築は難しいことが知られ、既存の物性予測の研究では、共通の条件で整理された大規模データが主に利用される(例えば、上のコンペティションなど)。小規模に限定されるデータ、例えば計算コストの膨大なシミュレーション値や実験データ、において、機械学習の利用は限定的であり、大きな研究課題の1つとなっている。

% 我々の研究チームはこのような少規模データに着目し研究を開始した。2021年度下半期は新手法提案への準備としてデータの収集に注力し研究を行った。機械学習分野や材料研究分野で用いられるオープンデータに加え、同学の工学部の研究チームへコンタクトし、スパコンスケールの計算資源を利用し得られた高価なシミュレーション値や実際の実験データに関してヒアリングを行い、データ収集を開始した。
% また、本部門で開発のすすめるmdxにおいては材料系研究への利用促進を行っており、本研究の中間報告として第20回ナノテクノロジー総合シンポジウムにて発表し、IEEE IC2E 2022への投稿論文にて材料系研究におけるクラウド基盤の利活用をまとめた。

% % 2021年度は主に、分野の調査と




% \begin{雑誌論文}{1}

% \bibitem{gnn-glass}
% Hayato Shiba, Masatoshi Hanai, Toyotaro Suzumura, and Takashi Shimokawabe, "BOTAN: BOnd TArgeting Network for prediction of slow glassy dynamics by machine learning relative motion." The Journal of Chemical Physics 158, no. 8, 084503, 2022.
% \end{雑誌論文}

% \begin{査読付}{3}

% \bibitem{xsig-limin-hanai}
% Limin Wang, Masatoshi Hanai, Toyotaro Suzumura, Shun Takashige, Kenjiro Taura, "On Data Imbalance in Molecular Property Prediction with Pre-training" xSIG 2023 (submitted)

% \bibitem{xsig-takashige-hanai}
% Shun Takashige, Masatoshi Hanai, Toyotaro Suzumura, Limin Wang, Kenjiro Taura, "Is Self-Supervised Pretraining Good for Extrapolation in Molecular Property Prediction?" xSIG 2023 (submitted)

% \bibitem{stgtrans-xiaohang}
% Xiaohang Xu, Toyotaro Suzumura, Jiawei Yong, Masatoshi Hanai, Chuang Yang, Hiroki Kanezashi, Renhe Jiang, Shintaro Fukushima, "Spatial-Temporal Graph Transformer for Next Point-of-Interest Recommendation", Machine Learning and Knowledge Discovery in Databases: European Conference, (ECML-PKDD), 2023 (submitted)

% \end{査読付}


\subsection{Research on efficient transformers and mining knowledge from unstructured data(Li Zihui)}

The first research branch is on efficient transformers. Efficient Transformers are popular for long sequence modeling due to their subquadratic memory and time complexity. However, Sparse Transformer, which improves efficiency by restricting self-attention to predefined sparse patterns, may compromise the expressiveness of the Transformer when important token correlations are distant. To overcome this limitation, we propose Diffuser, an efficient Transformer that incorporates all token interactions within one attention layer while maintaining low computation and memory costs. Diffuser leverages Attention Diffusion to expand the receptive field of sparse attention, allowing it to compute multi-hop token correlations based on all paths between corresponding disconnected tokens. We demonstrate the expressiveness of Diffuser as a universal sequence approximator and its ability to approximate full-attention using a graph expander property. Evaluations show that Diffuser outperforms state-of-the-art benchmarks in both expressiveness and efficiency for language modeling, image modeling, and Long Range Arena (LRA). We also propose a new method that improves the complexity of masked attention from $O(n^2)$ to $O(n)$ by decomposing it into local and global attention. Overall, Diffuser is a promising approach for long sequence modeling that combines the benefits of sparse attention and full-attention. 
\cite{feng2022diffuser}

Another research trial proposes a node neighborhood-enhanced framework for knowledge graph completion that enriches the head node information by modeling the head entity neighborhood from multiple hops using graph neural networks. The model also includes an additional edge link prediction task to improve KGC, which is evaluated on two public datasets and demonstrated to be simple yet effective. \cite{li2023nnkgc}

Lastly, a survey paper reviews the application of deep learning methods for Natural Language Processing (NLP) on electronic health records (EHRs). Recent advances in neural network and deep learning techniques have shown promising results in improving EHR analysis and outperforming traditional statistical and rule-based systems. The survey summarizes various neural NLP methods for EHR applications, including classification, prediction, word embeddings, extraction, generation, question answering, phenotyping, knowledge graphs, medical dialogue, multilinguality, and interpretability. \cite{li2022neural}

\begin{雑誌論文}{1}

\bibitem{li2022neural}
Irene Li, Jessica Pan, Jeremy Goldwasser, Neha Verma, Wai Pan Wong, Muhammed Yavuz Nuzumlalı, Benjamin Rosand, Yixin Li, Matthew Zhang, David Chang, R. Andrew Taylor, Harlan M. Krumholz, Dragomir Radev, "Neural Natural Language Processing for unstructured data in electronic health records: A review", Computer Science Review, volume 46, 2022.

\end{雑誌論文}

\begin{査読付}{1}

\bibitem{feng2022diffuser}
Aosong Feng and Irene Li and Yuang Jiang andRex  Ying, "Diffuser: Efficient Transformers with Multi-hop Attention Diffusion for Long Sequences", Proceedings of Thirty-Seventh AAAI Conference on Artificial Intelligence (AAAI), 2023.

\end{査読付}

\begin{発表}{1}

\bibitem{li2023nnkgc}
Zihui Li and Boming Yang and Toyotaro Suzumura, "NNKGC: Improving Knowledge Graph Completion with Node Neighborhoods", arXiv preprint, 2023.

\end{発表}

\subsection{Title(金刺 宏樹)}

\input{Kanezashi/ITCannual-list-Kanezashi}

% Ishikawa group
\subsection{高速知能システムの研究(石川正俊)}

 本研究室では、センサ情報処理における並列処理を基盤として、高速・リアルタイム性を有するセンサ情報処理を高度に実現し、高速知能システムとして実装する研究を行っている。具体的には、以下4つの分野での研究開発を行っている。また、各分野での新規産業開拓にも力を注ぎ、研究成果の技術移転、共同研究、事業化等を様々な形で積極的に推進している。

センサフュージョンの研究では、高速のセンサフィードバックに関する理論並びにシステムアーキテクチャの構築、その高速知能ロボットとしての実装、並びに高速性を活かした新規タスクの実現、特に、センサ情報に基づく人間機械協調システムの開発を行っている。

ダイナミックビジョンシステムの研究では、高速ビジョンや動的光学系に基づき運動対象の情報を適応的に取得する基礎技術の開発、特に、高速光軸制御や適応光学系の技術開発やトラッキング撮像に関する応用システムの開発を行っている。

システムビジョンデザインの研究では、高速三次元形状計測や高速質感計測など、並列処理に基づく高速画像処理技術 (理論、アルゴリズム、デバイス) の開発とその応用システムの実現を目指し、特に高速画像処理システムの開発や応用システムの開発を行っている。

アクティブパーセプションの研究では、実世界における新たな知覚補助技術並びにそれに基づく新しい対話の形の創出を目指し、特に各種高速化技術を用いた能動計測や能動認識を利用した革新的情報環境・ヒューマンインタフェイスの開発を行っている。

これらの研究では、原則的に並列かつリアルタイムの演算構造を有する現実の物理世界と同等の構造を工学的に実現することを目指しており、そのことにより、現実世界の理解を促すばかりでなく、従来のシステムをはるかに凌駕する性能を有する高速知能システムを生み出すことができ、結果として、まったく新しい情報システムを構築することが可能となる。


\begin{雑誌論文}{1}
\bibitem{01石川正俊01}
Masatoshi Ishikawa, Idaku Ishii, Hiromasa Oku, Akio Namiki, Yuji Yamakawa, and Tomohiko Hayakawa: Special Issue on High-Speed Vision and its Applications, Journal of Robotics and Mechatronics, Vol.34, No.5, p.911, 2022.

\bibitem{01石川正俊02}
Taku Senoo, Atsushi Konno, Yunzhuo Wang, Masahiro Hirano, Norimasa Kishi, and Masatoshi Ishikawa: Tracking of Overlapped Vehicles with Spatio-Temporal Shared Filter for High-Speed Stereo Vision, Journal of Robotics and Mechatronics, Vol.34, No.5, pp.1033-1042, 2022.

\bibitem{01石川正俊03}
Hyuno Kim, Yuji Yamakawa, and Masatoshi Ishikawa: Seamless Multiple-Target Tracking Method Across Overlapped Multiple Camera Views Using High-Speed Image Capture, Journal of Robotics and Mechatronics, Vol.34, No.5, pp.1043-1052, 2022.

\bibitem{01石川正俊04}
Masatoshi Ishikawa: High-Speed Vision and its Applications Toward High-Speed Intelligent Systems, Journal of Robotics and Mechatronics, Vol.34, No.5, pp.912-935 , 2022.

\end{雑誌論文}

\begin{発表}{1}

\bibitem{01石川正俊05}
Taku Senoo, Atsushi Konno, Yunzhuo Wang, Masahiro Hirano, Norimasa Kishi, and Masatoshi Ishikawa: Automotive Tracking with High-speed Stereo Vision Based on a Spatiotemporal Shared Filter, the 2022 26th International Conference on System Theory, Control and Computing (ICSTCC2022), Proceedings, pp.613-618, 2022.

\end{発表}

\begin{報道}{1}

\bibitem{01石川正俊06}
石川グループ研究室: 世界!オモシロ学者のスゴ動画祭3, NHK, 2022年7月8日, BSプレミアム

\end{報道}

\subsection{研究報告(早川 智彦)}

%  2021年度は主に1.高速画像処理技術によるモーションブラー補償、2.サーモカメラによる温度情報計測の定量化、3.人間の知覚情報の定量化の研究を実施した。全体を通した研究成果として、2件の雑誌論文(査読付)、3件の解説論文と1件の雑誌以外の査読付き論文を投稿し、4件の発表を行った。

% 1.高速画像処理技術によるモーションブラー補償

% インフラ点検に関する表面変状の撮像技術として、デフォーマブルミラーを用いたフォーカス調整とモーションブラー補償を同時に行う手法を提案することで、広い被写界深度と高速な移動の両方に対応する撮像環境に対応可能な技術を確立した。この成果を国際論文誌に投稿し、採択に至った。また、インフラ点検の関連技術をまとめ、複数の解説論文投稿や招待講演を行うことにより、技術の社会実装が円滑に進むよう努めた。

% 2.サーモカメラによる温度情報計測の定量化

% インフラ点検に関する内部変状の撮像技術として、サーモカメラによる撮像は有効であるが、同時に可視光画像の認識を行うことが困難であるため、両者に対応するマーカーを開発した。複数の素材の温度特性や可視光反射率を基に、両者のマーカーとしての見え方が最適化される素材の組み合わせを探求し、結果絶縁体と非絶縁体である黒い和紙と銅箔の組み合わせが最適であることを発見した。この成果を国際論文誌に投稿し、採択に至った。

% 3.人間の知覚情報の定量化

% 低遅延な映像のユーザへの影響を検証するため、高速カメラと高速プロジェクタを用い、遅延時間と対象の移動速度に応じてパフォーマンスが低下することを確認し、国内学会にて発表した。

\begin{招待論文}{1}

\bibitem{02早川智彦01}
早川智彦, 東晋一郎: 時速100km走行での覆工コンクリート高解像度変状検出手法, 建設機械, vol.58, no.4, pp.34-38, 2022.

\end{招待論文}

\begin{雑誌論文}{1}

\bibitem{02早川智彦02}
Yuriko Ezaki, Yushi Moko, Tomohiko Hayakawa, and Masatoshi Ishikawa: Angle of View Switching Method at High-Speed Using Motion Blur Compensation for Infrastructure Inspection, Journal of Robotics and Mechatronics,Vol. 34, No. 5, pp.985-996, 2022.

\bibitem{02早川智彦03}
Tomohiko Hayakawa, Yushi Moko, Kenta Morishita, Yuka Hiruma, Masatoshi Ishikawa: Tunnel Surface Monitoring System with Angle of View Compensation Function based on Self-localization by Lane Detection, Journal of Robotics and Mechatronics, Vol.34, No. 5, pp.997-1010, 2022.

\end{雑誌論文}

\begin{査読付}{1}

\bibitem{02早川智彦04}
Kairi Mine, Chika Nishimura, Tomohiko Hayakawa, Satoshi Yawata, Dai Watanabe, and Masatoshi Ishikawa: Migration correction technique using spatial information of neuronal images in fiber-inserted mouse under free-running behavior, Conf. on Neural Imaging and Sensing 2023, SPIE Photonics West BiOS/Proc. SPIE, Vol.1236522, pp.1236522: 1-1236522: 5, 2023.

\bibitem{02早川智彦05}
Yushan Ke, Yushi Moko, Yuka Hiruma, Tomohiko Hayakawa, Masatoshi Ishikawa: Silk-printed retroreflective markers for infrastructure-maintenance vehicles in tunnels, SPIE Smart Structures and Materials + Nondestructive Evaluation 2022 On Demand (Online), Paper 12046-18, 2022.

\bibitem{02早川智彦06}
Yushan Ke, Yushi Moko, Yuka Hiruma, Tomohiko Hayakawa, Elgueta Scarlet, Masatoshi Ishikawa: Silk-printed retroreflective markers for infrastructure-maintenance vehicles in curved tunnels, SPIE Smart Structures and Materials + Nondestructive Evaluation 2023, Paper 12483-40, 2023.

\end{査読付}

\begin{発表}{1}

\bibitem{02早川智彦07}
蛭間友香, 早川智彦, 石川正俊: 映像遅延および空間情報を制御可能な手の高速撮像・投影システムの構築, 第27回日本バーチャルリアリティ学会大会(vrsj2022)(札幌, 2022.9.14)/予稿集, 3F5-4, 2022.

\end{発表}


\subsection{研究報告(黄 守仁)}

%  本年度は主に人間協調ロボットの提案・実装、生産システムの知能化を目指した動的補償ロボットの開発、高速三次元計測によるロボット制御の検討を中心に研究活動を行った。

% 人間協調ロボットに関しては、人間の認知能力とロボットの高速・高精度な動作を相互補完的に組み合わせることを目指して、力覚提示によるヒューマンロボットインタラクションと人間の両腕同期運動現象を統合する手法の提案、タスク検証および被験者実験を行った。人間の両腕同期運動現象を定量的に分析し、得られた分析結果に基づいて適切な実験システムを構築した。10人ほどの被験者実験を実施し、両腕同期運動による力学提示を両腕の間に伝達する効果を検証した。また、企業との共同研究においても、人間ロボット共存型知能生産の実現を目指して、塗布など実応用を対象とした人間協調ロボットの設計・開発を行った。

% 次に、前年度に新規開発した3自由度動的補償モジュールを商用の産業ロボットに搭載し、従来の教示作業を必要としない産業用ロボットの知能化・自律化に向けて研究を推進した。また、企業との共同研究課題として、高速三次元計測を動的補償ロボット制御に統合するシステムの設計およびシミュレーションモデル構築などにも取り組んできた。その他に、前年度に開発した高周波電気刺激装置を用いて高周波外部フィードバック情報による人間の上腕に対する電気刺激制御の検証実験にも着手した。


% \begin{受賞}{1}
% \bibitem{03黄守仁01}
% 村上健一,黄守仁,石川正俊,山川雄司:高速ビジュアルフィードバックを用いた高速3次元位置補償システムの開発,第22回計測自動制御学会システムインテグレーション部門講演会(SI2021),講演会論文集,pp.1403-1405,優秀講演賞,2021.

% \end{受賞}

% \begin{査読付}{1}
% \bibitem{03黄守仁02}
% Mamoru Oka,Kenichi Murakami,Shouren Huang,Hirofumi Sumi,Masatoshi Ishikawa and Yuji Yamakawa:High-speed Manipulation of Continuous Spreading and Aligning a Suspended Towel-like Object,2022 IEEE/SICE International Symposium on System Integration,2022.

% \end{査読付}

% \begin{発表}{1}
% \bibitem{03黄守仁03}
% 村上 健一,黄 守仁,石川 正俊,山川 雄司:高速ビジュアルフィードバックを用いた高速3次元位置補償システムの開発,第22回計測自動制御学会システムインテグレーション部門講演会 (SI2021),講演会論文集,pp.1403-1405,2021.

% \bibitem{03黄守仁04}
% 岡衛,村上健一,黄守仁,角博文,石川正俊,山川雄司:面状柔軟物の展開に向けたコーナーの状態認識と把持動作計画,第39回日本ロボット学会学術講演会(RSJ2021),講演会論文集,3F1-01,2021.

% \bibitem{03黄守仁05}
% 上野永遠,黄守仁,石川正俊:上腕の一自由度回転運動に向けた高周波電気刺激フィードバック制御システムの構築,第39回日本ロボット学会学術講演会(RSJ2021),講演会論文集,RSJ2021AC2J1-01,2021

% \bibitem{03黄守仁06}
% 長谷川雄大,黄守仁,山川雄司,石川正俊:閉リンク機構を用いた動的補償モジュールの開発,第39回日本ロボット学会学術講演会(RSJ2021),講演会論文集,RSJ2021AC2D2-05,2021.

% \end{発表}


\subsection{ダイナミックビジョンシステムの研究開発(末石智大)}

 高速画像処理および高速光学系制御を用いた、動的検査技術とヒューマンインターフェースに関するダイナミックビジョンシステムの研究を実施した。

動的検査技術は、実世界の動的かつ複雑な現象を適応的にデータ化し、意味のある形で活用する技術となるものであり、昨年度に引き続き本年度もマイクロサッカードと呼ばれる眼球微振動などを対象として実施した。
被験者を拘束する負荷や時間効率の観点から、動的状態への検査技術の発展の期待は大きいと考えられる。頭部固定を必要とせずリラックスした状態の人間のマイクロサッカード検出に向けて、回転ミラーや液体可変焦点レンズなどの光学素子を高速に制御しつつ高解像度合焦画像計測を達成することで、ダイナミックビジョンシステムにおける動的検査への基礎技術の研鑽を進めている。
継続して開発を進めているマイクロサッカード計測システムの改良に加え、定量的評価のための動的眼球模型の両眼化を含む開発・改良、照明制御を伴う視線方向の高精度計測戦略、マイクロサッカード検知アルゴリズムを含む統合システム開発などの成果を実現した。

本年度は眼球運動以外にも、卓球ボールの回転運動をリアルタイムに計測するシステムや、液体可変焦点レンズを含むカメラ系の効率的な構成手法などにも取り組んだ。
特に球技スポーツであるテニスのライン判定を目的とした高速ビジョンと落下位置予測技術の融合システムの初期検討も実施し、本内容で国内学会において優秀講演賞を受賞した。
ヒューマンインターフェースに関しては、ベクター図形を描画可能なレーザー投影システムに不可視のバイナリ符号情報を埋め込んだ、自転車競技などへの活用が期待される、高速自己位置推定システムを開発し論文誌に掲載された。
光学系・照明系の適切な同期・校正手法を新たに開発することで、円に限定されない非対称なベクター図形を用いた情報提示と高速トラッキングの両立を可能とし、スポーツにおける複雑な身体運動のデータ化や即時フィードバックによる効率的な運動学習が期待される。

本年度は総じて、ヒトの眼球やボールの跳躍位置など瞬間的な動作に着目した高速センシングに関する新たな計測制御技術を、特に実世界応用へと繋がる成果として創出した。

% \begin{招待講演}{1}
% \bibitem{04末石智大01}
% 末石智大:高速光学系制御に基づくダイナミックビジョンシステムとその応用,第3回産業ロボット関連技術の標準化学術研究会,2022.

% \end{招待講演}

% \begin{招待論文}{1}
% \bibitem{04末石智大02}
% Yuri Mikawa,Tomohiro Sueishi,Yoshihiro Watanabe,and Masatoshi Ishikawa:Dynamic Projection Mapping for Robust Sphere Posture Tracking Using Uniform / Biased Circumferential Markers,2022 IEEE Conference on Virtual Reality and 3D User Interfaces (VR2022),(TVCG Invited)2022.

% \end{招待論文}

% \begin{受賞}{1}
% \bibitem{04末石智大03}
% 末石智大,石川正俊:手指高速トラッキングに向けた楕円群指輪マーカーの開発,第22回計測自動制御学会システムインテグレーション部門講演会(SI2021),講演会論文集,pp.1382-1387,優秀講演賞,2021.

% \end{受賞}

% \begin{雑誌論文}{1}

% \bibitem{04末石智大04}
% Yuri Mikawa,Tomohiro Sueishi,Yoshihiro Watanabe,and Masatoshi Ishikawa:Dynamic Projection Mapping for Robust Sphere Posture Tracking Using Uniform/Biased Circumferential Markers,IEEE Transaction on Visualization and Computer Graphics,1-1,2021 (Early Access).

% \bibitem{04末石智大05}
% 松本明弓,新田暢,末石智大,石川正俊:高速注視点推定を用いた広域高解像度投影システムの実現,計測自動制御学会論文集,Vol.58,No.1,pp.42-51,2022.

% \end{雑誌論文}

% \begin{査読付}{1}
% \bibitem{04末石智大06}
% Tomohiro Sueishi and Masatoshi Ishikawa:  Ellipses Ring Marker for High-speed Finger Tracking,The 27th ACM Symposium on Virtual Reality Software and Technology (VRST2021) (Osaka),Proceedings,Article No. 31,pp.1-5,2021.


% \bibitem{04末石智大07}
% Soichiro Matsumura,Tomohiro Sueishi,Shoji Yachida,and Masatoshi Ishikawa:Eye Vibration Detection Using High-speed Optical Tracking and Pupil Center Corneal Reflection,The 43rd Annual International Conference of the IEEE Engineering in Medicine and Biology Society (EMBC2021) (Virtual)/Proceedings,ThDT3.5,2021.

% \bibitem{04末石智大08}
% Ayumi Matsumoto,Tomohiro Sueishi,and Masatoshi Ishikawa:High-speed Gaze-oriented Projection by Cross-ratio-based Eye Tracking with Dual Infrared Imaging,2022 IEEE Conference on Virtual Reality and 3D User Interfaces Abstracts and Workshops (VRW2022),Proceedings,pp.594-595,2022.

% \bibitem{04末石智大09}
% Tomohiro Sueishi,Soichiro Matsumura,Shoji Yachida,and Masatoshi Ishikawa: Optical and Control Design of Bright-pupil Microsaccadic Artificial Eye,2022 IEEE/SICE International Symposium on System Integration (SII2022) Online,Proceedings,pp.760-765,2022.

% \end{査読付}

% \begin{発表}{1}
% \bibitem{04末石智大10}
% 末石智大,石川正俊:手指高速トラッキングに向けた楕円群指輪マーカーの開発,第22回計測自動制御学会システムインテグレーション部門講演会(SI2021),講演会論文集,pp.1382-1387,2021.

% \bibitem{04末石智大11}
% 末石智大,松村蒼一郎,谷内田尚司,石川正俊:マイクロサッカード高精度計測に向けた動的な明瞳孔眼球模型の開発,第22回計測自動制御学会システムインテグレーション部門講演会(SI2021),講演会論文集,pp.2011-2016,2021.

% \bibitem{04末石智大12}
% 松村蒼一郎,末石智大,井上満晶,谷内田尚司,石川正俊:光学系制御撮影下の角膜反射法によるマイクロサッカード検出高精度化の検討,第22回計測自動制御学会システムインテグレーション部門講演会(SI2021),講演会論文集,pp.2025-2028,2021.

% \bibitem{04末石智大13}
% 三河祐梨,末石智大,石川正俊:球体姿勢に対応した回転相殺テクスチャの高速投影の残像効果による一軸回転可視化法の提案,第26回日本バーチャルリアリティ学会大会(VRSJ2021),論文集,2D2-5,2021.

% \bibitem{04末石智大14}
% 三河祐梨,末石智大,渡辺義浩,石川正俊:VarioLight2円周マーカを用いた球体への広域かつ遮蔽に頑健なダイナミックプロジェクションマッピング,第27回画像センシングシンポジウム(SSII2021),講演論文集 IS1-25,2021.

% \end{発表}


\subsection{高速3次元形状計測技術の評価と応用(宮下令央)}

 本年度は昨年度開発した高速3次元形状計測技術の評価、および関連する高速画像処理システムの研究・発表を行った。
 新たに開発した高速3次元形状計測技術によって、従来手法よりも高解像度かつ高精度な3次元形状計測が1,000fpsを超える高速性と低遅延性を維持したまま実現できることを確認し、成果を国内外で発表した。

また、高速3次元形状計測の関連研究として、従来のテレセントリック光学系による画像検査では困難な長尺の棒材の寸法検測を非テレセントリック光学系によって実現するシステムや、高速3次元形状計測と高速法線計測を統合して高密度かつ高精度かつ高速な3次元形状計測を実現するシステム、さらにイメージセンサに高速画像処理を行うプロセッサを並置したビジョンチップを用いて小型化を実現した高速光軸制御システムについて、論文誌において発表を行った。

さらに、人間の錯覚を利用して仮想的な運動を付加するダイナミックプロジェクションマッピング技術をコンピュータグラフィックスに応用し、アニメーションのレンダリングを高速化する研究を行い、被験者実験を通して得た人間の知覚特性に関する知見について国際学会において発表を行った。
今後は高速3次元形状計測技術を発展させ、質感計測やダイナミックプロジェクションマッピングへの応用を進めていく予定である。

また、解説論文の執筆、研究会の運営や査読を通して学会への貢献を続けている。

\begin{招待論文}{1}

\bibitem{05宮下令央01}
宮下 令央, 末石 智大, 田畑 智志, 早川 智彦, 石川 正俊: 高速ビジョンが拓くダイナミックプロジェクションマッピング技術, 電子情報通信学会 通信ソサイエティマガジン B-plus, 解説論文, No.64, pp.275-284, 2023. 

\end{招待論文}

\begin{雑誌論文}{1}

\bibitem{05宮下令央02}
Leo Miyashita, Masatoshi Ishikawa: Portable High-speed Optical Gaze Controller with Vision Chip, Journal of Robotics and Mechatronics(JRM), Vol.34, No.5, pp.1133-1140, 2022.

\bibitem{05宮下令央03}
Leo Miyashita, Yohta Kimura, Satoshi Tabata, Masatoshi Ishikawa: High-speed Depth-normal Measurement and Fusion Based on Multiband Sensing and Block Parallelization, Journal of Robotics and Mechatronics (JRM), Vol.34, No.5, pp.1111-1121, 2022.

\bibitem{05宮下令央04}
Leo Miyashita, Masatoshi Ishikawa: Real-Time Inspection of Rod Straightness and Appearance by Non-Telecentric Camera Array, Journal of Robotics and Mechatronics (JRM), Vol.34, No.5, pp.975-984, 2022.

\bibitem{05宮下令央05}
Yunpu Hu, Leo Miyashita, and Masatoshi Ishikawa: Differential Frequency Heterodyne Time-of-Flight Imaging for Instantaneous Depth and Velocity Estimation. ACM Trans. Graph. Vol.42, No.1, Article 9, pp.9, 2022.

\end{雑誌論文}

\begin{査読付}{1}

\bibitem{05宮下令央06}
宮下 令央, 田畑 智志, 石川 正俊: パラレルバスパターンによる高速低遅延3次元形状計測, 計測自動制御学会, 第39回 センシングフォーラム, 1B1-1, 予稿集 pp.49-54, 2022.

\bibitem{05宮下令央07}
Leo Miyashita, Satoshi Tabata, Masatoshi Ishikawa: High-speed and Low-latency 3D Sensing with a Parallel-bus Pattern, International Conference on 3D Vision(3DV2022), 2022.

\bibitem{05宮下令央08}
Leo Miyashita, Kentaro Fukamizu, Yuki Kubota, Tomohiko Hayakawa, Masatoshi Ishikawa: Real-time animation display based on optical illusion by overlaid luminance changes, SPIE Optical Architectures for Displays and Sensing in Augmented, Virtual, and Mixed Reality(AR, VR, MR)IV, Oral, paper 12449-8, 2023.

\end{査読付}


\subsection{小型高速三次元スキャナの開発および可変光学系による技術拡張に関する研究(田畑智志)}

 本年度は主に小型高速三次元スキャナの開発を昨年度に引き続き実施するとともに、計測システムの可変焦点化による計測技術の向上方法の検討と、形状計測の高精度高解像化やダイナミックプロジェクションマッピング技術の広範囲化に関する実験を実施した。

高速三次元スキャン技術は、運動する物体の形状取得やロボットにおける外界認識・インタラクションなど、高速性・リアルタイム性が要求される応用において重要である。昨年度に引き続き、小型ユニットを用いた1,000fpsでの高速小型三次元スキャナの開発を行い、国内学会での口頭発表や招待論文、報道による周知を行った。
リアルタイムのモデル生成と運動補正を組みあわせることで1,000fpsの高速性を維持したまま、1辺10cmの立方体に収まるサイズで片手に持ってスキャン可能な計測システムを実現している。また、この技術をさらに発展させるため、さらなる統合アプローチの検討を進めている。

さらに、三次元形状計測について、計測距離範囲・精度を向上させるため高速可変焦点レンズの導入を進め予備実験を実施した。また、位相シフト法における計測精度の向上や高解像化に際して課題となる点を精査し、それぞれの課題を解決するためのアルゴリズムの開発を進めている。

また、ダイナミックプロジェクションマッピングに関しては、対象の三次元的な情報を計測・利用するシステムに対する回転ミラーを用いたトラッキングアプローチの拡張を進めており、実際の応用を通じた実験を行っている。



% \begin{雑誌論文}{1}

% \bibitem{06田畑智志01}
% Hongjin Xu,Lihui Wang,Satoshi Tabata,Yoshihiro Watanabe,and Masatoshi Ishikawa:Extended depth-of-field projection method using a high-speed projector with a synchronized oscillating variable-focus lens,Applied Optics,Vol.60,Issue 13,pp.3917-3924,2021.

% \end{雑誌論文}

% \begin{査読付}{1}
% \bibitem{06田畑智志02}
% Yuping Wang,Senwei Xie,Lihui Wang,Hongjin Xu,Satoshi Tabata,and Masatoshi Ishikawa:ARSlice: Head-Mounted Display Augmented with Dynamic Tracking and Projection,The 10th international conference on Computational Visual Media (CVM 2022),2022(accepted).

% \end{査読付}



\section{データ科学研究部門 成果要覧}
%\begin{招待講演}{1}

\bibitem{kobayashi3-1}
Hill Hiroki Kobayashi, mdx: A Cloud Platform for Supporting Data Science and Cross-Disciplinary Research Collaborations, the Nepal JSPS Alumni Association (NJAA), hosted its 7th Symposium, 29 November, 2022.

\bibitem{kobayashi3-2}
小林博樹, mdx を用いたフィールドデータの蓄積と公開の可能性, 東京大学大学院農学生命科学研究科附属演習林 生態水文学研究所100周年記念式典, 8 Dec, 2022.

\bibitem{ykuga37056192}
空閑洋平, クラウド環境におけるネットワークインタフェースの高機能化, 第155回 システムソフトウェアとオペレーティング・システム研究会, 26 May, 2022.


\bibitem{fugaku} 鈴村豊太郎「夢の形~未来のコンピュータ~」(パネリスト)、スーパーコンピュータ「富岳」第2回成果創出加速プログラムシンポジウム「富岳百景」、2022年12月21日
\bibitem{jsse1}鈴村豊太郎,「データ活用社会創成プラットフォームmdxおよび 大規模グラフニューラルネットワーク」, 第34回CCSEワークショップ「原子力材料研究開発におけるDX推進の現状と将来:原子力材料研究開発の革新と新展開」, 2023年2月24日
\bibitem{jsse2}鈴村豊太郎,「人工知能の最先端研究に迫る ~大規模グラフニューラルネットワークの世界へ」, 東京大学柏キャンパス一般公開2022、特別講演会、2022年10月22日
\bibitem{rakuten}Toyotaro Suzumura, “How Will Data and AI Change the World?”,  Rakuten Optimism Conference, Tokyo, Japan, September 29, 2022
\bibitem{rccs}Toyotaro Suzumura,  “Large-Scale Graph Neural Networks for Real-World Industrial Applications”, The 5th R-CCS International Symposium, Kobe Japan, February 7, 2023
\bibitem{france}Toyotaro Suzumura,  “Large-Scale Graph Neural Networks for Real-World Industrial Applications”, International Workshop “HPC challenges for new extreme scale application” held by French Alternative Energies and Atomic Energy Commission, Paris, France, March 6, 2023
\bibitem{barcelona}Toyotaro Suzumura,  “Large-Scale Graph Neural Networks for Real-World Industrial Applications”, Barcelona Supercomputing Center, Barcelona, Spain, March 10, 2023
\end{招待講演}

\begin{招待論文}{1}


\bibitem{02早川智彦01}
早川智彦, 東晋一郎: 時速100km走行での覆工コンクリート高解像度変状検出手法, 建設機械, vol.58, no.4, pp.34-38, 2022.


\bibitem{05宮下令央01}
宮下 令央, 末石 智大, 田畑 智志, 早川 智彦, 石川 正俊: 高速ビジョンが拓くダイナミックプロジェクションマッピング技術, 電子情報通信学会 通信ソサイエティマガジン B-plus, 解説論文, No.64, pp.275-284, 2023. 


\bibitem{06田畑智志01}
田畑智志, 渡辺義浩, 石川正俊: 小型高速三次元スキャナの研究開発と将来展望, 日本ロボット工業会機関紙『ロボット』, 271号, pp.45-47, 2023.

\bibitem{ykuga41534619}
杉木章義, 空閑洋平, 竹房あつ子, 藤原一毅, 合田憲人, 中村遼, 塙敏博, 鈴村豊太郎, 宮本大輔, 田浦健次朗, 伊達進, 建部修見, データ活用社会創成に向けた基盤ソフトウェア環境の構築, 学術情報処理研究, 26, 1, pp1-9, Dec, 2022.


\end{招待論文}

\begin{受賞}{1}


\bibitem{03黄守仁01}
黄守仁,村上健一, 石川正俊: 対象の事前情報必要としない動的塗布応用に向けたロボットの実現, 第23回計測自動制御学会システムインテグレーション部門講演会(SI2022), 講演会論文集, pp.999-1001, SI2022 優秀講演賞, 2022.


\bibitem{04末石智大01}
栃岡陽麻里, 末石智大, 石川正俊: 球技スポーツの着地痕跡判定に向けた高速ビジョンを用いた落下位置予測, 第23回計測自動制御学会システムインテグレーション部門講演会(SI2022), 講演会論文集, pp.2089-2092, 優秀講演賞, 2022.


\end{受賞}

\begin{著書}{1}



\bibitem{fl-book}
Toyotaro Suzumura, Yi Zhou, Ryo Kawahara, Nathalie Baracaldo, Heiko Ludwig,
"Federated Learning for Collaborative Financial Crimes Detection",
Ludwig, H., Baracaldo, N. (eds) Federated Learning. Springer, Cham. https://doi.org/10.1007/978-3-030-96896-0\_20

\bibitem{jsai-gnn}
鈴村 豊太郎, 金刺 宏樹, 華井 雅俊, グラフニューラルネットワークの広がる活用分野, 人工知能, 2023, 38 巻, 2 号, p. 139-148, 公開日 2023/03/02, Online ISSN 2435-8614, Print ISSN 2188-2266, https://doi.org/10.11517/jjsai.38.2\_139

\end{著書}

\begin{雑誌論文}{1}

\bibitem{01石川正俊01}
Masatoshi Ishikawa, Idaku Ishii, Hiromasa Oku, Akio Namiki, Yuji Yamakawa, and Tomohiko Hayakawa: Special Issue on High-Speed Vision and its Applications, Journal of Robotics and Mechatronics, Vol.34, No.5, p.911, 2022.

\bibitem{01石川正俊02}
Taku Senoo, Atsushi Konno, Yunzhuo Wang, Masahiro Hirano, Norimasa Kishi, and Masatoshi Ishikawa: Tracking of Overlapped Vehicles with Spatio-Temporal Shared Filter for High-Speed Stereo Vision, Journal of Robotics and Mechatronics, Vol.34, No.5, pp.1033-1042, 2022.

\bibitem{01石川正俊03}
Hyuno Kim, Yuji Yamakawa, and Masatoshi Ishikawa: Seamless Multiple-Target Tracking Method Across Overlapped Multiple Camera Views Using High-Speed Image Capture, Journal of Robotics and Mechatronics, Vol.34, No.5, pp.1043-1052, 2022.

\bibitem{01石川正俊04}
Masatoshi Ishikawa: High-Speed Vision and its Applications Toward High-Speed Intelligent Systems, Journal of Robotics and Mechatronics, Vol.34, No.5, pp.912-935 , 2022.


\bibitem{02早川智彦02}
Yuriko Ezaki, Yushi Moko, Tomohiko Hayakawa, and Masatoshi Ishikawa: Angle of View Switching Method at High-Speed Using Motion Blur Compensation for Infrastructure Inspection, Journal of Robotics and Mechatronics,Vol. 34, No. 5, pp.985-996, 2022.

\bibitem{02早川智彦03}
Tomohiko Hayakawa, Yushi Moko, Kenta Morishita, Yuka Hiruma, Masatoshi Ishikawa: Tunnel Surface Monitoring System with Angle of View Compensation Function based on Self-localization by Lane Detection, Journal of Robotics and Mechatronics, Vol.34, No. 5, pp.997-1010, 2022.


\bibitem{03黄守仁02}
Kenichi Murakami, Shouren Huang, Masatoshi Ishikawa, and Yuji Yamakawa: Fully Automated Bead Art Assembly for Smart Manufacturing Using Dynamic Compensation Approach, J. Robot. Mechatron., Vol.34, No.5, pp.936-945, 2022. 

\bibitem{03黄守仁03}
Shouren Huang, Kenichi Murakami, Masatoshi Ishikawa, Yuji Yamakawa: Robotic Assistance Realizing Peg-and-Hole Alignment by Mimicking the Process of an Annular Solar Eclipse, Journal of Robotics and Mechatronics, Vol.34 No.5, pp.946-955, 2022. 


\bibitem{04末石智大02}
井上満晶, 末石智大, 松村蒼一郎, 谷内田尚司, 細井利憲, 石川正俊: 非接触マイクロサッカード検出に向けた高速追跡を用いた眼球運動検出システム, 生体医工学, Vol. 60, No. 6, pp.170-174, 2022.

\bibitem{04末石智大03}
Tomohiro Sueishi, Ryota Nishizono, and Masatoshi Ishikawa: EmnDash:A Robust High-Speed Spatial Tracking System Using a Vector-Graphics Laser Display with M-Sequence Dashed Markers, J. Robot. Mechatron., Vol.34, No.5, pp.1085-1095, 2022.


\bibitem{05宮下令央02}
Leo Miyashita, Masatoshi Ishikawa: Portable High-speed Optical Gaze Controller with Vision Chip, Journal of Robotics and Mechatronics(JRM), Vol.34, No.5, pp.1133-1140, 2022.

\bibitem{05宮下令央03}
Leo Miyashita, Yohta Kimura, Satoshi Tabata, Masatoshi Ishikawa: High-speed Depth-normal Measurement and Fusion Based on Multiband Sensing and Block Parallelization, Journal of Robotics and Mechatronics (JRM), Vol.34, No.5, pp.1111-1121, 2022.

\bibitem{05宮下令央04}
Leo Miyashita, Masatoshi Ishikawa: Real-Time Inspection of Rod Straightness and Appearance by Non-Telecentric Camera Array, Journal of Robotics and Mechatronics (JRM), Vol.34, No.5, pp.975-984, 2022.

\bibitem{05宮下令央05}
Yunpu Hu, Leo Miyashita, and Masatoshi Ishikawa: Differential Frequency Heterodyne Time-of-Flight Imaging for Instantaneous Depth and Velocity Estimation. ACM Trans. Graph. Vol.42, No.1, Article 9, pp.9, 2022.


\bibitem{06田畑智志02}
Lihui Wang, Satoshi Tabata, Hongjin Xu, Yunpu Hu, Yoshihiro Watanabe, and Masatoshi Ishikawa: Dynamic depth-of-field projection mapping method based on a variable focus lens and visual feedback, Optics Express, Vol.31, Issue 3, pp.3945-3953, 2023.

\bibitem{06田畑智志03}
Hao Xu, Satoshi Tabata, Haowen Liang, Lihui Wang, and Masatoshi Ishikawa: Accurate measurement of virtual image distance for near-eye displays based on auto-focusing, Applied Optics, Vol.61, Issue 30, pp.9093-9098, 2022.

\bibitem{06田畑智志04}
Yuping Wang, Senwei Xie, Lihui Wang, Hongjin Xu, Satoshi Tabata, and Masatoshi Ishikawa, ARSlice: Head-Mounted Display Augmented with Dynamic Tracking and Projection, Journal of Computer Science and Technology, Vol.37, Issue 3, pp.666-679, 2022.


% \bibitem{gnn-glass}
% Hayato Shiba, Masatoshi Hanai, Toyotaro Suzumura, and Takashi Shimokawabe, "BOTAN: BOnd TArgeting Network for prediction of slow glassy dynamics by machine learning relative motion." The Journal of Chemical Physics 158, no. 8, 084503, 2022.

\bibitem{osada.pccp.24.21705}
Wataru Osada, Shunsuke Tanaka, Kozo Mukai, Mitsuaki Kawamura, YoungHyun Choi, Fumihiko Ozaki, Taisuke Ozaki and Jun Yoshinobu, 
"Elucidation of the atomic-scale processes of dissociative adsorption and spillover of hydrogen on the single atom alloy catalyst Pd/Cu(111)",
Physical Chemistry and Chemical Physics 24, 36, 21705-21713, 2022.

\bibitem{hirai.jacs.144.17857}
Daigorou Hirai, Keita Kojima, Naoyuki Katayama, Mitsuaki Kawamura, Daisuke Nishio-Hamane, and Zenji Hiroi,
"Linear Trimer Molecule Formation by Three-Center-Four-Electron Bonding in a Crystalline Solid RuP",
Journal of the American Chemical Society 144, 39, 17857-17864, 2022.

\bibitem{koshoji.prm.6.114802}
Ryotaro Koshoji, Masahiro Fukuda, Mitsuaki Kawamura, Taisuke Ozaki,
"Prediction of quaternary hydrides based on densest ternary sphere packings",
Physical Review Materials 6, 11, 114802.1-9, 2022.

\bibitem{kobayashi1-1}
Wenjing Li, Haoran Zhang, Ryosuke Shibasaki,  Jinyu Chen and Hill Hiroki Kobayashi,  "Mining individual significant places from historical trajectory data", Handbook of Mobility Data Mining, Jan, 2023.

\bibitem{kobayashi1-2}
Wenjing Li, Haoran Zhang, Ryosuke Shibasaki,  Jinyu Chen and Hill Hiroki Kobayashi,  "Mobility pattern clustering with big human mobility data", Handbook of Mobility Data Mining, Jan, 2023.

\bibitem{kobayashi1-3}
Wenjing Li, Haoran Zhang, Jinyu Chen, Peiran Li, Yuhao Yao, Xiaodan Shi,  Mariko Shibasaki, Hill Hiroki Kobayashi, Xuan Song and Ryosuke Shibasaki,  "Metagraph-Based Life Pattern Clustering With Big Human Mobility Data", IEEE Transactions on Big Data, Feb, 2023.

\bibitem{JIANG1-1}
Renhe Jiang, Zekun Cai, Zhaonan Wang, Chuang Yang, Zipei Fan, Quanjun Chen, Xuan Song, and Ryosuke Shibasaki, "Predicting Citywide Crowd Dynamics at Big Events: A Deep Learning System", ACM Trans. Intell. Syst. Technol. (TIST), 13, 2, Article 21, April 2022.
\bibitem{JIANG1-2}
Zipei Fan, Chuang Yang, Zhiwen Zhang, Xuan Song, Yinghao Liu, Renhe Jiang, Quanjun Chen, and Ryosuke Shibasaki, "Human Mobility-based Individual-level Epidemic Simulation Platform", ACM Trans. Spatial Algorithms Syst. (TSAS), 8, 3, Article 19, September 2022.
\bibitem{JIANG1-3}
Chuang Yang, Zhiwen Zhang, Zipei Fan, Renhe Jiang, Quanjun Chen, Xuan Song, Ryosuke Shibasaki, "EpiMob: Interactive Visual Analytics of Citywide Human Mobility Restrictions for Epidemic Control", IEEE Transactions on Visualization and Computer Graphics (TVCG), 2022.
\bibitem{JIANG1-4}
Yudong Tao, Chuang Yang, Tianyi Wang, Erik Coltey, Yanxiu Jin, Yinghao Liu, Renhe Jiang, Zipei Fan, Xuan Song, Ryosuke Shibasaki, Shu-Ching Chen, Mei-Ling Shyu, Steven Luis, "A Survey on Data-Driven COVID-19 and Future Pandemic Management", ACM Computing Surveys (CSUR), 2022.
\bibitem{JIANG1-5}
Yongkang Li, Zipei Fan, Du Yin, Renhe Jiang, Jinliang Deng, Xuan Song, "HMGCL: Heterogeneous Multigraph Contrastive Learning for LBSN Friend Recommendation", World Wide Web (WWW), 2022.
\bibitem{JIANG1-6}
Renhe Jiang, Quanjun Chen, Zekun Cai, Zipei Fan, Xuan Song, Kota Tsubouchi, and Ryosuke Shibasaki, “Will You Go Where You Search? A Deep Learning Framework for Estimating User Search-and-Go Behavior”, Neurocomputing, Volume 472, 2022, Pages 338-348.
\bibitem{JIANG1-7}
Jinliang Deng, Xiusi Chen, Renhe Jiang, Xuan Song, Ivor W. Tsang, "A Multi-view Multi-task Learning Framework for Multi-variate Time Series Forecasting", IEEE Transactions on Knowledge and Data Engineering (TKDE), 2022.


\bibitem{feta}
Bastos, Anson, Abhishek Nadgeri, Kuldeep Singh, Hiroki Kanezashi, Toyotaro Suzumura, and Isaiah Onando Mulang, "How Expressive Are Transformers in Spectral Domain for Graphs?", Transactions on Machine Learning Research (TMLR) ISSN 2835-8856, Journal of Machine Learning Research, 2022.

\bibitem{Uesaka2023HCII}
Uesaka L, Khatiwada AP, Shimotoku D, Parajuli LK, Pandey MR, Kobayashi HH (2023). Applications of Bioacoustics Human Interface System for Wildlife Conservation in Nepal. Accepted for publication in Human Computer Interaction (HCI) International 2023.


\bibitem{li2022neural}
Irene Li, Jessica Pan, Jeremy Goldwasser, Neha Verma, Wai Pan Wong, Muhammed Yavuz Nuzumlalı, Benjamin Rosand, Yixin Li, Matthew Zhang, David Chang, R. Andrew Taylor, Harlan M. Krumholz, Dragomir Radev, "Neural Natural Language Processing for unstructured data in electronic health records: A review", Computer Science Review, volume 46, 2022.

\end{雑誌論文}

\begin{査読付}{1}


\bibitem{02早川智彦04}
Kairi Mine, Chika Nishimura, Tomohiko Hayakawa, Satoshi Yawata, Dai Watanabe, and Masatoshi Ishikawa: Migration correction technique using spatial information of neuronal images in fiber-inserted mouse under free-running behavior, Conf. on Neural Imaging and Sensing 2023, SPIE Photonics West BiOS/Proc. SPIE, Vol.1236522, pp.1236522: 1-1236522: 5, 2023.

\bibitem{02早川智彦05}
Yushan Ke, Yushi Moko, Yuka Hiruma, Tomohiko Hayakawa, Masatoshi Ishikawa: Silk-printed retroreflective markers for infrastructure-maintenance vehicles in tunnels, SPIE Smart Structures and Materials + Nondestructive Evaluation 2022 On Demand (Online), Paper 12046-18, 2022.

\bibitem{02早川智彦06}
Yushan Ke, Yushi Moko, Yuka Hiruma, Tomohiko Hayakawa, Elgueta Scarlet, Masatoshi Ishikawa: Silk-printed retroreflective markers for infrastructure-maintenance vehicles in curved tunnels, SPIE Smart Structures and Materials + Nondestructive Evaluation 2023, Paper 12483-40, 2023.


\bibitem{05宮下令央06}
宮下 令央, 田畑 智志, 石川 正俊: パラレルバスパターンによる高速低遅延3次元形状計測, 計測自動制御学会, 第39回 センシングフォーラム, 1B1-1, 予稿集 pp.49-54, 2022.

\bibitem{05宮下令央07}
Leo Miyashita, Satoshi Tabata, Masatoshi Ishikawa: High-speed and Low-latency 3D Sensing with a Parallel-bus Pattern, International Conference on 3D Vision(3DV2022), 2022.

\bibitem{05宮下令央08}
Leo Miyashita, Kentaro Fukamizu, Yuki Kubota, Tomohiko Hayakawa, Masatoshi Ishikawa: Real-time animation display based on optical illusion by overlaid luminance changes, SPIE Optical Architectures for Displays and Sensing in Augmented, Virtual, and Mixed Reality(AR, VR, MR)IV, Oral, paper 12449-8, 2023.


\bibitem{06田畑智志05}
Yuping Wang, Senwei Xie, Lihui Wang, Hongjin Xu, Satoshi Tabata, Masatoshi Ishikawa: Head-Mounted Display Augmented with Dynamic Tracking and Projection, The 10th international conference on Computational Visual Media(CVM 2022), 2022.


% \bibitem{xsig-limin-hanai}
% Limin Wang, Masatoshi Hanai, Toyotaro Suzumura, Shun Takashige, Kenjiro Taura, "On Data Imbalance in Molecular Property Prediction with Pre-training" xSIG 2023 (submitted)

% \bibitem{xsig-takashige-hanai}
% Shun Takashige, Masatoshi Hanai, Toyotaro Suzumura, Limin Wang, Kenjiro Taura, "Is Self-Supervised Pretraining Good for Extrapolation in Molecular Property Prediction?" xSIG 2023 (submitted)

% \bibitem{stgtrans-xiaohang}
% Xiaohang Xu, Toyotaro Suzumura, Jiawei Yong, Masatoshi Hanai, Chuang Yang, Hiroki Kanezashi, Renhe Jiang, Shintaro Fukushima, "Spatial-Temporal Graph Transformer for Next Point-of-Interest Recommendation", Machine Learning and Knowledge Discovery in Databases: European Conference, (ECML-PKDD), 2023 (submitted)

\bibitem{kobayashi2-1}
Daisuk\'e Shimotoku, Tian Yuan, Laxmi Kumar Parajuli and Hill Hiroki Kobayashi, "Participatory Sensing Platform Concept for Wildlife Animals in the Himalaya Region, Nepal", Proceedings of 2022 International Conference on Human-Computer Interaction (HCII 2022), 2022.  

\bibitem{kobayashi2-2}
Qaiser Anwar, Muhammad Hanif, Daisuk\'e Shimotoku and  Hiroki Kobayashi, "Driver awareness collision/proximity detection system for heavy vehicles based on deep neural network", International Symposium on Intelligent Unmanned Systems and Artificial Intelligence (SIUSAI 2022) , 2022.  

\bibitem{kobayashi2-3}
Usman Haider, Muhammad Hanif, Laxmi Kumar Parajuli, Hill Hiroki Kobayashi, Daisuk\'e Shimotoku, Ahmar Rashid and Sonia Safeer, "Bioacoustics signal classification using hybrid feature space with machine learning",  International Conference on Computer and Automation Engineering (ICCAE 2023), 2023.  

\bibitem{ykuga40356877}
空閑洋平, 中村遼, 遠隔会議システムの計測データを用いた広域ネットワーク品質計測, インターネットと運用技術シンポジウム論文集, 2022, Dec, 2022.

\bibitem{ykuga39987672}
Shu Anzai, Masanori Misono, Ryo Nakamura, Yohei Kuga, Takahiro Shinagawa, Towards isolated execution at the machine level, Proceedings of the 13th ACM SIGOPS Asia-Pacific Workshop on Systems, 23 Aug, 2022.

\bibitem{ykuga36919054}
Yukito Ueno, Ryo Nakamura, Yohei Kuga, Hiroshi Esaki, Pktpit: separating routing and packet transfer for fast and scalable software routers, Proceedings of the 37th ACM/SIGAPP Symposium on Applied Computing, 25 Apr, 2022.

\bibitem{JIANG2-1}
Renhe Jiang, Zhaonan Wang, Jiawei Yong, Puneet Jeph, Quanjun Chen, Yasumasa Kobayashi,
Xuan Song, Shintaro Fukushima, Toyotaro Suzumura, "Spatio-Temporal Meta-Graph Learning for Traffic Forecasting", Proceedings of Thirty-Seventh AAAI Conference on Artificial Intelligence (AAAI), 2023.
\bibitem{JIANG2-2}
Renhe Jiang, Zekun Cai, Zhaonan Wang, Chuang Yang, Zipei Fan, Quanjun Chen, Kota Tsubouchi, Xuan Song, Ryosuke Shibasaki, "Yahoo! Bousai Crowd Data: A Large-Scale Crowd Density and Flow Dataset in Tokyo and Osaka", Proc. of 2022 IEEE International Conference on Big Data (BigData), 2022.
\bibitem{JIANG2-3}
Xinchen Hao, Renhe Jiang, Jiewen Deng, Xuan Song, "The Impact of COVID-19 on Human Mobility: A Case Study on New York", Proc. of 2022 IEEE International Conference on Big Data (BigData), 2022.
\bibitem{JIANG2-4}
Zheng Dong, Quanjun Chen, Renhe Jiang, Hongjun Wang, Xuan Song, Hao Tian, "Learning Latent Road Correlations from Trajectories", Proc. of 2022 IEEE International Conference on Big Data (BigData), 2022.
\bibitem{JIANG2-5}
Zipei Fan, Xiaojie Yang, Wei Yuan, Renhe Jiang, Quanjun Chen, Xuan Song, Ryosuke Shibasaki, "Online Trajectory Prediction for Metropolitan Scale Mobility Digital Twin", Proc. of 30th ACM SIGSPATIAL International Conference on Advances in Geographic Information Systems (SIGSPATIAL), 2022. 
\bibitem{JIANG2-6}
Haoyuan Ma, Mintao Zhou, Xiaodong Ouyang, Du Yin, Renhe Jiang, Xuan Song, "Forecasting Regional Multimodal Transportation Demand with Graph Neural Networks: An Open Dataset", Proc. of 2022 IEEE International Intelligent Transportation Systems Conference (ITSC), 2022.
\bibitem{JIANG2-7}
Qi Cao, Renhe Jiang, Chuang Yang, Zipei Fan, Xuan Song, Ryosuke Shibasaki, "MepoGNN: Metapopulation Epidemic Forecasting with Graph Neural Networks", Proceedings of the European Conference on Machine Learning and Principles and Practice of Knowledge Discovery in Databases (ECML PKDD), 2022. 
\bibitem{JIANG2-8}
Hangli Ge, Lifeng Lin, Renhe Jiang, Takashi Michikata, Noboru Koshizuka, "Multi-weighted Graphs Learning for Passenger Count Prediction on Railway Network," 2022 IEEE 46th Annual Computers, Software, and Applications Conference (COMPSAC), 2022.


\bibitem{sigir}
Md Mostafizur Rahman, Daisuke Kikuta, Satyen Abrol, Yu Hirate, Toyotaro Suzumura, Pablo Loyola, Takuma Ebisu and Manoj Kondapaka, "Exploring 360-Degree View of Customers for Lookalike Modeling",  SIGIR'23 (The 46th International ACM
SIGIR Conference on Research and Development in Information
Retrieval) 


\bibitem{aaai-deft}
Anson Bastos, Abhishek Nadgeri, Kuldeep Singh, Toyotaro Suzumura, Manish Singh,
"Learnable Spectral Wavelets on Dynamic Graphs to Capture Global Interactions"
The 37th AAAI Conference on Artificial Intelligence (AAAI), 2023.

\bibitem{aaai-megacrn}
Renhe Jiang, Zhaonan Wang, Jiawei Yong, Puneet Jeph, Quanjun Chen, Yasumasa Kobayashi, Xuan Song, Shintaro Fukushima, Toyotaro Suzumura,
"Spatio-Temporal Meta-Graph Learning for Traffic Forecasting"
37th AAAI Conference on Artificial Intelligence (AAAI), 2023.

\bibitem{kg-kp}
Anson Bastos, Kuldeep Singh, Abhishek Nadgeri, Johannes Hoffart, Toyotaro Suzumura, Manish Singh,
"Can Persistent Homology provide an efficient alternative for Evaluation of Knowledge Graph Completion Methods?"
In proceedings of The Web Conference (WWW), 2023.


\bibitem{gqsm}
Chinthaka Weerakkody, Miyuru Dayarathna, Sanath Jayasena, Toyotaro Suzumura
"Guaranteeing Service Level Agreements for Triangle Counting via Observation-based Admission Control Algorithm", IEEE 15th International Conference on Cloud Computing (CLOUD 2022), 2022.

\bibitem{mdx}
Toyotaro Suzumura, Akiyoshi Sugiki, Hiroyuki Takizawa, Akira Imakura, Hiroshi Nakamura, Kenjiro Taura, Tomohiro Kudoh, Toshihiro Hanawa, Yuji Sekiya, Hiroki Kobayashi, Shin Matsushima, Yohei Kuga, Ryo Nakamura, Renhe Jiang, Junya Kawase, Masatoshi Hanai, Hiroshi Miyazaki, Tsutomu Ishizaki, Daisuke Shimotoku, Daisuke Miyamoto, Kento Aida, Atsuko Takefusa, Takashi Kurimoto, Koji Sasayama, Naoya Kitagawa, Ikki Fujiwara, Yusuke Tanimura, Takayuki Aoki, Toshio Endo, Satoshi Ohshima, Keiichiro Fukazawa, Susumu Date, Toshihiro Uchibayashi,
"mdx: A Cloud Platform for Supporting Data Science and Cross-Disciplinary Research Collaborations", 2022 IEEE Intl Conf on Cloud and Big Data Computing, (CBDCom), 2022.

\bibitem{botan}
Hayato Shiba, Masatoshi Hanai, Toyotaro Suzumura, and Takashi Shimokawabe, "BOTAN: BOnd TArgeting Network for prediction of slow glassy dynamics by machine learning relative motion." The Journal of Chemical Physics 158, no. 8, 084503, 2022.

\bibitem{eth-gnn}
Hiroki Kanezashi, Toyotaro Suzumura, Xin Liu, Takahiro Hirofuchi,
"Ethereum Fraud Detection with Heterogeneous Graph Neural Networks"
28TH ACM SIGKDD Conference on Knowledge Discovery and Data Mining (KDD'22), Workshop on Mining and Learning with Graphs, 2022.

\bibitem{stgtrans}
Xiaohang Xu, Toyotaro Suzumura, Jiawei Yong, Masatoshi Hanai, Chuang Yang, Hiroki Kanezashi, Renhe Jiang, Shintaro Fukushima, "Spatial-Temporal Graph Transformer for Next Point-of-Interest Recommendation", Machine Learning and Knowledge Discovery in Databases: European Conference, (ECML-PKDD), 2023 (submitted)


\bibitem{xsig-limin}
Limin Wang, Masatoshi Hanai, Toyotaro Suzumura, Shun Takashige, Kenjiro Taura, "On Data Imbalance in Molecular Property Prediction with Pre-training" xSIG 2023 (cross-disciplinary workshop on computing Systems, Infrastructures, and programminG)  (submitted)

\bibitem{xsig-takashige}
Shun Takashige, Masatoshi Hanai, Toyotaro Suzumura, Limin Wang, Kenjiro Taura, "Is Self-Supervised Pretraining Good for Extrapolation in Molecular Property Prediction?" xSIG 2023 (cross-disciplinary workshop on computing Systems, Infrastructures, and programminG) (submitted)


% \bibitem{suzumura-sc2021}
% Venkatesan T. Chakaravarthy, Shivmaran S. Pandian, Saurabh Raje, Yogish Sabharwal, Toyotaro Suzumura, Shashanka Ubaru, 
% "Efficient scaling of dynamic graph neural networks". SC2021(The International Conference for High Performance Computing, Networking, Storage, and Analysis)

% \bibitem{suzumura-smds21}
% Shilei Zhang, Toyotaro Suzumura, Li Zhang, "DynGraphTrans: Dynamic Graph Embedding via Modified Universal Transformer Networks for Financial Transaction Data", IEEE SMDS 2021 (International Conference on Smart Data Services) 

\bibitem{feng2022diffuser}
Aosong Feng and Irene Li and Yuang Jiang andRex  Ying, "Diffuser: Efficient Transformers with Multi-hop Attention Diffusion for Long Sequences", Proceedings of Thirty-Seventh AAAI Conference on Artificial Intelligence (AAAI), 2023.

\end{査読付}

\begin{公開}{1}


\end{公開}

\begin{特許}{1}


\end{特許}

\begin{発表}{1}


\bibitem{01石川正俊05}
Taku Senoo, Atsushi Konno, Yunzhuo Wang, Masahiro Hirano, Norimasa Kishi, and Masatoshi Ishikawa: Automotive Tracking with High-speed Stereo Vision Based on a Spatiotemporal Shared Filter, the 2022 26th International Conference on System Theory, Control and Computing (ICSTCC2022), Proceedings, pp.613-618, 2022.


\bibitem{02早川智彦07}
蛭間友香, 早川智彦, 石川正俊: 映像遅延および空間情報を制御可能な手の高速撮像・投影システムの構築, 第27回日本バーチャルリアリティ学会大会(vrsj2022)(札幌, 2022.9.14)/予稿集, 3F5-4, 2022.


\bibitem{03黄守仁04}
黄守仁, 村上健一, 石川正俊: 対象の事前情報必要としない動的塗布応用に向けたロボットの実現, 第23回計測自動制御学会システムインテグレーション部門講演会(SI2022), 講演会論文集, pp.999-1001, 2022.

\bibitem{03黄守仁05}
村上健一,黄守仁,石川正俊,山川雄司: 動的補償を用いたビーズピッキング, 第40回日本ロボット学会学術講演会 (RSJ2022), 予稿集, 4C1-05, 2022.

\bibitem{03黄守仁06}
Shouren Huang, Yongpeng Cao, Kenichi Murakami, Masatoshi Ishikawa,Yuji Yamakawa: Bimanual Coordination Protocol for the Inter-Limb Transmission of Force Feedback, 第40回日本ロボット学会学術講演会 (RSJ2022), 予稿集, 2C1-05, 2022.


\bibitem{04末石智大04}
末石智大, 井上満晶, 谷内田尚司, 石川正俊: 照明制御に基づく動的眼球高速トラッキングによる明瞳孔微振動画像計測, 動的画像処理実利用化ワークショップ2023(DIA2023), 講演論文集, pp.133-136, IS1-20, 2023.

\bibitem{04末石智大05}
栃岡陽麻里, 末石智大, 石川正俊: 球技スポーツの着地痕跡判定に向けた高速ビジョンを用いた落下位置予測, 第23回計測自動制御学会システムインテグレーション部門講演会(SI2022), 講演会論文集, pp.2089-2092, 2022.

\bibitem{04末石智大06}
末石智大, 石川正俊: 縞状同心円パターンを用いた可変焦点制御系のカメラ校正手法, 第23回計測自動制御学会システムインテグレーション部門講演会(SI2022), 講演会論文集, pp.2093-2097, 2022.

\bibitem{04末石智大07}
井上満晶, 末石智大, 松村蒼一郎, 谷内田尚司, 細井利憲, 石川正俊: 非接触マイクロサッカード検出に向けた高速追跡を用いた眼球運動検出システム, 生体医工学シンポジウム2022(オンライン), 予稿・抄録集, p.13, 1A-11, 2022.

\bibitem{04末石智大08}
末石智大, 石川正俊: 高速光学系制御と対称的ドットマーカによる卓球回転実時間計測, 第40回日本ロボット学会学術講演会 (RSJ2022), 予稿集, 2B1-08, 2022.


\bibitem{06田畑智志06}
田畑智志, 渡辺義浩, 石川正俊: 小型高速三次元スキャナの開発, 第40回日本ロボット学会学術講演会(RSJ2022), 予稿集, 2B1-07, 2022.


\bibitem{ykuga41835081}
中村遼, 空閑洋平, 複数コネクションを用いる高速なscpの実装, 研究報告システムソフトウェアとオペレーティング・システム(OS), Feb, 2023.

\bibitem{ykuga41835070}
空閑洋平, 中村遼, ソフトウェアメモリを用いたNVMeコマンドのキャプチャ, 研究報告システムソフトウェアとオペレーティング・システム(OS), Feb, 2023.


\bibitem{sms}
脇 聡志, 鈴村 豊太郎, 金刺 宏樹, 小林 秀, "強化学習によるマッチング数を最大化するジョブ推薦システム." 第37回人工知能学会全国大会 (2023),  一般社団法人 人工知能学会, (2023年6月発表予定)


\bibitem{li2023nnkgc}
Zihui Li and Boming Yang and Toyotaro Suzumura, "NNKGC: Improving Knowledge Graph Completion with Node Neighborhoods", arXiv preprint, 2023.

\end{発表}

\begin{特記}{1}


\end{特記}

\begin{報道}{1}


\bibitem{01石川正俊06}
石川グループ研究室: 世界!オモシロ学者のスゴ動画祭3, NHK, 2022年7月8日, BSプレミアム


\bibitem{06田畑智志07}
田畑智志: 毎秒1000回撮像で立体計測 小型高速3Dスキャナー ロボアームの高速操作向け 東大など開発, 日刊工業新聞, 2022年9月27日.


\end{報道}


最後に全員分の成果をマージする。

\end{document}

